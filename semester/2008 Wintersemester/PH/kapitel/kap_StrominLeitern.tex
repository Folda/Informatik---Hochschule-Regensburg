\chapter{Strom in Leitern}
\section{Elektrische Leitung in Metallen}
\textalign{Metall:}{
kristallener Festk�rper}
$\ra$ Atome untereinander zu Gitter verbunden (meist bei Metallen, kovalente Bindung)\\
01.12.2008-IMG-phys-1\\
\textalign{im Metall:}{nicht alle Au�en Elektronen zur Bindungen notwendig}
\renewcommand{\labelitemi}{$\rightarrow$}
\begin{itemize}
\item Abgabe von nicht ben�tigten Elektronen
\item abgegebene $e^-$ frei im Gitter beweglich (''Elektronengas'')
\end{itemize}
\renewcommand{\labelitemi}{$\bullet$}
bei Anlegen eines �u�eren Feldes:\\
freie $e^-$ bewegen sich im Feld\\
$\ra$ Strom\\
\textalign{Elektrostatik:}{$\epsilon = 0$ in  Leitern}\\
\textalign{genauer:}{$e^-$ bewegen sich  so lange, bis $\epsilon = 0$}\\
01.12.2008-IMG-phys-2\\
falls Ladungstr�ger abflie�en und auf anderer Seite wieder zuflie�en k�nnen:\\
$\epsilon_{\mbox{gegen}}$ aufbauen unm�glich $\ra \epsilon \neq 0$ in Leiter

Nachtragen von Jurij -- 04.12.2008 (Donnerstag)

Strom $\equals$ bewegten Ladungstr�gern\\
08.12.2008-IMG-phys-1
\[J = \frac{dQ}{dt}\]
\[J \propto U\]
\[I = \frac{1}{R} \mal U \mbox{ ohmsche Gesetz}\]
\[\Ra R = \frac{U}{I}, G = \frac{1}{R}\]

\section{Widerstandsschaltungen}
\subsection{Reihenschaltungen}
08.12.2008-IMG-phys-2 $\equals$ 08.12.2008-IMG-phys-3\\
\begin{description}
\item[Strom] $I_i$ durch Widerst�nde gleich (keine Verzweigung)\\
		\fbox{$I_i = I$}
\item[Spannung] $U_i$ addieren sich zur Gesamtspannung $U$ auf\\
		$\Ra$ \fbox{$U = \sumx{i = 1}{N}{U_i}$}
		\[\Ra U = R \mal I \mbox{ f�r } R_{\mbox{Ges}} \mbox{ und } R_i\]
		\[\sumx{i = 1}{N}{U_i} = R_{\mbox{Ges}} \mal I = \sumx{i = 1}{N}{R_i \mal I_i} = I \mal \sumx{i = 1}{N}{R_i}\]
		\[I \mal \sumx{i = 1}{N}{R_i} = I \mal R_{\mbox{Ges}}\]
		\fbox{$R_{\mbox{Ges}} = \sumx{i = 1}{N}{R_i}$} Widerst�nde addieren sich
\end{description}

\subsubsection{Beispiel: Spannungsteiler}
\textalign{08.12.2008-IMG-phys-4}{$I_1 = I_2 = I$ da Reihenschaltung\\
$I = \frac{U}{R_{\mbox{Ges}}} = \frac{U}{R_1 + R_2}$\\
$U_2 = R_2 \mal I_2 = R_2 \mal I$
$\Ra$ \fbox{$U_2 = U \mal \frac{R_2}{R_1 + R_2}$}}

\subsection{Parallelschaltung}
08.12.2008-IMG-phys-5 $\equals$ 08.12.2008-IMG-phys-6\\
\begin{description}
\item[Spannungen] $U_i$ an Widerst�nden $R_i$ sind gleich\\
		$\Ra$ \fbox{$U_i = U$}
\item[Str�me] $I_i$ addieren sich zum Gesamtstrom $I$\\
		\fbox{$I = \sumx{i = 1}{N}{I_i}$}\\
		\[I = \frac{1}{R} \mal U\]
		\[I = \sumx{i = 1}{N}{I_i} = \sumx{i = 1}{N}{\frac{1}{R_i} \mal U_i} = U \mal \sumx{i = 1}{N}{\frac{1}{R_i}} = U \mal \frac{1}{R_{\mbox{Ges}}}\]
		\textalign{$\Ra$}{\fbox{$\dfrac{1}{R_{\mbox{Ges}}} = \sumx{i = 1}{N}{\dfrac{1}{R_i}}$}\\
		\fbox{$G_{\mbox{Ges}} = \sumx{i = 1}{N}{G_i}$} $G = \frac{1}{R}$ Leitwerte addieren sich}\\
		\textalign{$\ra$}{Gesamtwiderstand einer Parallelschaltung ist immer kleiner als der kleinste Einzelwiderstand}
\end{description}

\subsubsection{Beispiel: Br�ckenschaltung}
08.12.2008-IMG-phys-7\\
Ersatzschaltung: $\left(R_1 + R_2\right) \Vert \left(R_3 + R_4\right)$
\[U_D = U_2 - U_4\]
\begin{align*}
	U_2 = R_2 \mal I_{12} ~~ &U_4 = R_4 \mal I_{34}\\
	I_{12} = \frac{U}{R_1 + R_2} ~~ &I_{34} = \frac{U}{R_3 + R_4}\\
	U_2 = U \mal \frac{R_2}{R_1 + R_2} ~~ &U_4 = U \mal \frac{R_4}{R_3 + R_4}
\end{align*}
$\Ra$ \fbox{$U_D = U \left(\dfrac{R_2}{R_1 + R_2} - \dfrac{R_4}{R_3 + R_4}\right)$}

\section{Arbeit und Leistung}
Die Arbeit $W$, um die Ladung $Q$ durch Potentialdifferenz (Spannung) $U$ zu bewegen, (siehe Kapitel \vref{sec:PotentialeundSpannungen-2})
\[W = Q \mal U\]
f�r $I = \mbox{ const}$ $I = \frac{Q}{t} \Ra Q = I \mal t$\\
$\Ra$ \fbox{$W = U \mal I \mal t$} $[w] = VAs = \dots = J = Ws$\\
$\mbox{Leistung} = \frac{\mbox{Arbeit}}{\mbox{Zeit}}$
\[P = \frac{W}{t}\]
\fbox{$P = \dfrac{U \mal I \mal t}{t} = U \mal I$} $[P] = VA = W$ ''Watt''\\
$U = I \mal R$ $\ra$ \fbox{$P = I^2 \mal R$}\\
$I = \frac{U}{R}$ $\ra$ \fbox{$P = \frac{U^2}{R}$}

\textalign{Beispiel:}{Energiekosten PC 24/7 (st�ndig an)\\
\textalign{Gegeben:}{Leistungsaufnahme $P = 130 W$\\
Preis $K = 0,1885 \frac{\EUR}{kW h}$}\\
\textalign{L�sung:}{Energieaufnahme pro Jahr:
\begin{align*}
	W = P \mal t &= 130W \mal 365 \mal 24 h\\
	&= 1139 kWh
\end{align*}
Kosten pro Jahr:
\begin{align*}
	W \mal K &= 1139 kWh \mal 0,1885 \frac{\EUR}{kWh}\\
	&= 214,70 \EUR
\end{align*}}}

\section{MOS-Feldeffekttransistor (MOS-FET)}
\begin{description}
\item[MOS] Metal Oxid Silicone
\item[Transistor] \begin{itemize}
						\item analog elektrisch steuerbarer Widerstand\\
						\item digital elektrisch steuerbarer Schalter
						\end{itemize}
\end{description}

\subsection{Schalt Symbol}
11.12.2008-IMG-phys-1\\
$\left.\begin{array}{l}
\mbox{Gate}\\
\mbox{Bulk}
\end{array}\right\}$ Steuereing�nge\\
$\left.\begin{array}{l}
\mbox{Drain}\\
\mbox{Source}
\end{array}\right\}$ Schalter

\subsection{Aufbau (Querschnitt)}
11.12.2008-IMG-phys-2\\
\textalign{$Si$ ist}{rein \ra nicht leitf�hig\\
dotiert \ra leitf�hig}

$G$, $B$ Steuereing�nge\\
\subsection{\texorpdfstring{$U_{GB} = 0$}{U mit Index GB gleich 0}}
\begin{itemize}
\item keine freien Ladungstr�ger zwischen $D$ und $S$
\end{itemize}
\renewcommand{\labelitemi}{$\rightarrow$}
\begin{itemize}
\item kein Stromfluss von $D$ nach $S$ m�glich
\item Transistor sperrt 11.12.2008-IMG-phys-3
\end{itemize}
\renewcommand{\labelitemi}{$\bullet$}
\subsection{\texorpdfstring{$U_{GB} > 0$}{U mit Index GB gr��er 0}}
Vereinzelte Ladungstr�ger aus $Si$ werden durch Coulombkr�fte in den Kanal an die Grenzschicht zum Oxid gezogen.
\renewcommand{\labelitemi}{$\rightarrow$}
\begin{itemize}
\item viele Ladungstr�ger im Kanal
\item leitf�higer Kanal
\item Stromfluss von $D$ nach $S$ ist m�glich
\item Transistor leitet 11.12.2008-IMG-phys-4
\end{itemize}
\renewcommand{\labelitemi}{$\bullet$}
meistens $S$ und $B$ kurzgeschlossen (�ndert an Funktion nichts)
\textalign{$\ra$}{$U_{GB} = U_{GS}$ (Steuerspannung)}

\subsection{Vorteile}
\begin{itemize}
\item leistungslos steuerbar (Oxid ist isolierend)
\item hohe Schaltfrequenzen m�glich (GHz-Bereich)
\item mikroelektronisches Bauelement
		\renewcommand{\labelitemii}{$\rightarrow$}
		\begin{itemize}
		\item kleinste Abmessungen (zur Zeit $\approx$ 100nm)
		\item hohe Packungsdichten (CPU $\approx$ 100 Mio Transistoren)
		\item MOS-FET ist DAS Grundbau Element der Digitaltechnik und somit der Informationstechnologie
		\end{itemize}
		\renewcommand{\labelitemii}{$\bullet$}
\end{itemize}

\section{Logikschaltungen}
\textalign{Digitaltechnik:}{nur '0' oder '1'\\
\textalign{Beispiel TTL-Pegel:}{$0V$ $\equals$ '0'\\
$5V$ $\equals$ '1'}\\
$\ra$ Transistor $\equals$ Schalter}

\subsection{NMOS-Inverter}
\subsubsection{Inverter}
Wahrheitstabelle:\\
\begin{tabular}{c|c}
input & output\\
$x$ & $y$
\\\hline
$0$ & $1$
\\$1$ & $0$
\end{tabular} $\Ra$
\begin{tabular}{c|c}
$U_{in}$ & $U_{out}$
\\\hline
$0V$ & $5V$
\\$5V$ & $0V$
\end{tabular}

\subsubsection{Schaltung}
11.12.2008-IMG-phys-5\\
\textalign{$U_{in} = 0V$}{$\equals$ x = '0'\\
$U_{GS} = 0V$
\renewcommand{\labelitemi}{$\rightarrow$}
\begin{itemize}
\item Transistor sperrt (Schalter offen) 11.12.2008-IMG-phys-6
\item $I = 0 \mal A$
\item $U_R = I \mal R = 0V$
\item $U_{cc} = U_R + U_{out}$
\item $U_{out} = U_{cc} = 5V$ $\equals$ '1'
\end{itemize}
\renewcommand{\labelitemi}{$\bullet$}}

\textalign{$U_{in} = 5V$}{$\equals$ '1'\\
$U_{GS} = 5V > 0V$
\renewcommand{\labelitemi}{$\rightarrow$}
\begin{itemize}
\item Transistor leitet (geschlossener Schalter) 11.12.2008-IMG-phys-7
\item Kurzschluss $U_{out}$ zu $0V$ $\ra$ $U_{out} = 0V$
\end{itemize}
\renewcommand{\labelitemi}{$\bullet$}
$U_{out} = 0V$ $\equals$ '0'}

$\Ra$ Inverter

\textalign{Nachteil:}{Stromfluss �ber R\\
bei $U_{in} = 5V$
\renewcommand{\labelitemi}{$\Rightarrow$}
\begin{itemize}
\item $P = I^2 \mal R$
\item Leistung = W�rme an $R$
\end{itemize}
\renewcommand{\labelitemi}{$\bullet$}}

\subsection{CMOS-Inverter}
C: Complementary

\subsubsection{2 Typen von MOS-FETS}
\textalign{u-FET:}{\textkreis{-} im Kanal
\renewcommand{\labelitemi}{$\rightarrow$}
\begin{itemize}
\item leitet bei $U_{GS} > 0$\\
		sperrt bei $U_{GS} = 0$
\end{itemize}
\renewcommand{\labelitemi}{$\bullet$}}

\textalign{p-FET:}{\textkreis{+} im Kanal
\renewcommand{\labelitemi}{$\rightarrow$}
\begin{itemize}
\item leitet bei $U_{GS} < 0$\\
		sperrt bei $U_{GS} = 0$
\end{itemize}
\renewcommand{\labelitemi}{$\bullet$}}

11.12.2008-IMG-phys-8

\subsubsection{$U_{in} = 0V$ $\equals$ '0'}
\textalign{u-FET:}{$U_{GS} = U_G - U_S = 0V - 0V = 0V$\\
$\ra$ sperrt}

\textalign{p-FET:}{$U_{GS} = U_G - U_S = 0V - 5V = -5V$\\
$\ra$ leitet}

\renewcommand{\labelitemi}{$\Rightarrow$}
\begin{itemize}
\item Kurzschluss $U_{out} \leftrightarrow U_{cc}$
\item $U_{out} = U_{cc} = 5V$ $\equals$ '1'
\end{itemize}
\renewcommand{\labelitemi}{$\bullet$}

\subsubsection{$U_{in} = 5V$ $\equals$ '1'}
\textalign{u-FET:}{$U_{GS} = U_G - U_S = 5V - 0V = 5V$\\
$\ra$ leitet}

\textalign{p-FET:}{$U_{GS} = U_G - U_S = 5V - 5V = 0V$\\
$\ra$ sperrt}

\renewcommand{\labelitemi}{$\Rightarrow$}
\begin{itemize}
\item Kurzschluss $U_{out} \leftrightarrow 0V$
\item $U_{out} = 0V$ $\equals$ '0'
\end{itemize}
\renewcommand{\labelitemi}{$\bullet$}

\textalign{Vorteil:}{Kein Stromfluss zwischen $U_{cc}$ und 0V (GND) im statischen Zustand $\ra$ kein Leistungseinbruch}

11.12.2008-IMG-phys-9

\subsection{NAND-Gatter}
Wahrheitstabelle:

\begin{tabular}{cc|c}
A & B & y\\
\hline
0 & 0 & 1\\
0 & 1 & 1\\
1 & 0 & 1\\
1 & 1 & 0
\end{tabular}

Aufbau:

15.12.2008-IMG-phys-1

\textalign{$U_{out} = 0V$?}{
nur f�r beide nFETs leitend
\[\left.\begin{array}{ll}
	\ra &U_A = 5V\\
	\mbox{und } &U_B = 5V
\end{array}\right\} \Ra \mbox{ pFETs sperren}\]

15.12.2008-IMG-phys-2 $\Ra$ $U_{out} = 0V$}

\textalign{$U_{out} = 5V$?}{
mindestens ein pFET muss leiten
\[\left.\begin{array}{ll}
\ra & U_A = 0V\\
\mbox{oder } & U_B = 0V
\end{array}\right\} \Ra \mbox{ mindestens ein uFET sperrt}\]

15.12.2008-IMG-phys-3}

\subsection{NOR-Gatter}
Wahrheitstabelle:

\begin{tabular}{cc|c}
A & B & y\\
0 & 0 & 1\\
0 & 1 & 0\\
1 & 0 & 0\\
1 & 1 & 0
\end{tabular}

Aufbau:

15.12.2008-IMG-phys-4

\textalign{$U_{out} = 5V$?}{
nur f�r beide pFETs leitend
\[\left.\begin{array}{ll}
&U_A = 0V\\
\mbox{und } & U_B = 0V
\end{array}\right\} \Ra \mbox{ uFETs sperren}\]

15.12.2008-IMG-phys-2}

\textalign{$U_{out} = 0V$?}{
f�r mindestens einen uFET leitend
\[\left.\begin{array}{ll}
& U_A = 5V\\
\mbox{oder } & U_B = 5V
\end{array}\right\} \Ra \mbox{ pFET sperren}\]}

$\Ra$ NOR-Gatter

15.12.2008-IMG-phys-5 15.12.2008-IMG-phys-6

\textalign{$\Ra$}{jeder logische Funktion aus Inverter, NAND und NOR aufbauen}

\textalign{Beispiel:}{AND = NAND + Inverter}

TODO: Itemize mit RA
\begin{itemize}
\item Rechenoperationen ($+, - , \mal, \div$) auf logische Funktionen �berf�hrbar ($\ra$ boolsche Algebra)
\item Ablaufsteuerungen sind logischer Funktionen
\item CPU: komplexe Verschaltung von Logik Gattern in Form von Transistoren
\end{itemize}
