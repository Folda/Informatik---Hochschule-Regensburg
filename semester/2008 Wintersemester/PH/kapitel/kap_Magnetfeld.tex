\chapter{Magnetfelder}
\textalign{genauer:}{magnetostatische Felder (daher. zeitlich nicht ver�nderbare Magnetfelder)}

\section{Eigenschaften von Magnetfeldern}
\textalign{Beobachtung:}{Magnete �ben Kr�fte aufeinander aus}

TODO: Itemize mit ra
\begin{itemize}
\item Beschreibung der Kraftwirkung durch Felder\\
		(wie bei Coulombkraft \ra elektrostatische Feld)
\end{itemize}

\begin{itemize}
\item Richtung der Feldlinien gibt Richtung der Kraft an\\
		Dichte der Feldlinien gibt St�rke der Kraft/Feldes an

\item magnetische Feldlinien sind immer in sich geschlossen (''quellenfreies Wirbelfeld'')
		$\ra$
\end{itemize}

\begin{itemize}
\item Kraft $\ra$ Feld
\item Feldlinien stets geschlossen
\item Nord- und S�dpol treten stets gemeinsam auf
\item gleiche Pole sto�en sich ab
		18.12.2008-IMG-phys-1

\item ungleiche Pole ziehen sich an
		18.12.2008-IMG-phys-2

\item Feldlinien treten am Nordpol aus einem Magneten aus, und am S�dpol wieder ein.
\item Gr��en:
		\begin{itemize}
		\item Magnetische Flussdichte $B$: Kraft- und Spannungsberechnung
		\item Magnetische Feldst�rke $H$: Stromberechnung\\
				\fbox{$B = \mu H$}\\
		\end{itemize}
		$\mu$ Permeabilit�t (Materialkonstante)\\
		\fbox{$\mu = \mu_r \mu_0$}\\
		Vakuum: $\mu_r = 1$\\
		\fbox{$\mu_0 = 4 \pi \znr{-7} \dfrac{Vs}{Am}$}

\item Einheiten:\\
		$[B] = \frac{N}{Am} = \frac{Vs}{m^2} = T$ (Tesla)\\
		$[H] = \frac{A}{m}$

\item Ursache f�r Magnetfelder:\\
		18.12.2008-DIA-phys-3\\
		Richtung des Magnetfelds:\\
		Rechte-Hand-Regel\\
		D�ume in Richtung der technischen Stromrichtung, gekr�mmter Finger in Richtung des Magnetfeldes

\item St�rke (Betrag) des Magnetfelds\\
		Durchflutungsgesetz\\
		\fbox{$\oint_{S} \vec{H} mal d\vec{s} = I_{\mbox{eing}}$} 18.12.2008-IMG-phys-5\\
		18.12.2008-IMG-phys-6 18.12.2008-IMG-phys-7 in Tafelebene hinein\\
		$I = \oint_{S} \vec{H} \mal d\vec{s} \underbrace{=}_{\vec{H} \vert \vert d \vec{s}} \oint_{S} H ds = \underbrace{=}_{H = \mbox{ const auf Kurve } S} \underbrace{H \oint_S ds}_{\mbox{L�nge der Kurve}}$\\
		$I = H \mal 2 \pi r$\\
		\fbox{$H = \dfrac{I}{2 \pi} \dfrac{1}{r}$}\\
		DIA-phys-8\\
		$B = \mu H$\\
		hier: Vakuum $\ra$ $\mu_r = 1$\\
		\fbox{$B = \mu_0 H = \dfrac{\mu_0}{Q \pi} I \mal \dfrac{1}{r}$}\\
		Vorgehensweise (analog zum Gau�schen Satz):
		\begin{itemize}
		\item Magnetfeld qualitativ zeichnen
		\item Kurve $S$ passend zur Feldgeometrie w�hlen $(\vec{H} \vert \vert d \vec{s})$
		\item $I$ eingeschlossen bestimmen (eine Richtung $\textcircled{+}$, andere Richtung $\textcircled{-}$)
		\item Skalarprodukt und Linienintegral vereinfacht sich $\Ra H (r)$
		\item $B (r) = \mu H (r)$
		\end{itemize}
\end{itemize}

\subsection{Lorenzkraft}
Str�me (bewegte Ladungen) verursachen ein Magnetfeld ($\equals$ Kraftwirkung)\\
\textalign{$\Ra$}{auf bewegte Ladungen im Magnetfeld wirkt eine Kraft (Lorenzkraft)}\\
18.12.2008-IMG-phys-8\\
$\vec{B}$ (homogenes Magnetfeld)
\begin{align*}
	q &\mbox{ bewegte Ladung}\\
	\vec{F_L} &\mbox{ Lorenzkraft}
\end{align*}

Richtung von $\vec{F_L}$:\\
Drei-Finger-Regel\\
rechte Hand
\begin{align*}
	\vec{V}: & \mbox{ Daumen}\\
	\vec{B}: & \mbox{ Zeigefinger}\\
	\vec{F_L}: & \mbox{ Mittelfinger (auf positive Ladung)}
\end{align*}

\[\vec{F_L} = q \mal \left(\vec{v} \times \vec{B}\right) \mbox{ allgemeine vektorielle Form}\]
22.12.2008-IMG-phys-1 $\begin{array}{ll}v_x \neq 0 & v_y = v_z = 0\\B_y \neq 0 & B_x = B_z = 0\end{array}$
\[\vektor{F_x}{F_y}{F_z} = q \mal \vektor{v_x}{0}{0} \times \vektor{0}{B_y}{0} = q \mal \vektor{0 - 0}{0 - 0}{v_x B_y - 0}\]
\[F_{Lx} = F_{Ly} = 0\]
\fbox{$F_{LZ} = q \mal v_x \mal B_y$} f�r $\vec{v}$ senkrecht $\vec{B}$ TODO: senkrecht zeichen einf�gen

\paragraph{Stromdurchflossener Leiter im homogenen Magnetfeld:}

$I = $ const. 22.12.2008-IMG-phys-2
\begin{itemize}
\item Ladung $q$ braucht Zeit $t = \frac{l}{v}$ um durch den Leiter zu flie�en.
\item In der ZEit $t$ fliet eine Ladungsmenge $Q = I \mal t = I \mal \frac{l}{v}$ durch den Leiter.\\
		$\Ra \vec{v} = \frac{I}{Q} \mal \vec{l}$ $\leftarrow$ Richtung des technischen Stroms
		\begin{align*}
		\vec{F_L} &= Q \mal \left(\vec{v} \times \vec{B}\right)\\
		&= I \mal \left(\vec{l} \times \vec{B}\right)
		\end{align*}
		\fbox{$\vec{F_L} = I \mal \left(\vec{l} \times \vec{B}\right)$}\\
		\textalign{\fbox{$F = I \mal l_x \mal B_y$}}{f�r $\vec{l}$ senkrecht $\vec{B}$ TODO: senkrecht zeichen einf�gen\\
		$\begin{array}{l}
		l_y = l_z = 0, l_x \neq 0\\
		B_x = B_z = 0, B_y \neq 0
		\end{array}$}
\end{itemize}

\paragraph{Stromdurchflossene Leiterschleife im homogenen Magnetfeld:}

22.12.2008-IMG-phys-3 22.12.2008-IMG-phys-4\\
$\Ra$ Drehmoment auf Leiterschleife\\
\textalign{Anwendung:}{Elektromotor\\
Drehspulenmesswerk (Strommessung)}

\section{Freie Elektronen im Magnetfeld}
\textalign{$\ra$}{Ablenkung von $e^-$ durch Lorenzkraft}
22.12.2008-IMG-phys-5\\
$\vec{F_L}$ senkrecht $\vec{v}$ TODO: senkrecht zeichen einf�gen\\
%Arritemize
\begin{itemize}
\item Lorenzkraft kann nur die Richtung $\vec{v}$ �ndern, nicht den Betrag.
\item $e^-$ bewegen sich im homogenen Magnetfeld unter dem Einfluss der Lorenzkraft auf einer Kreisbahn mit dem Radius $r$.
\end{itemize}
Kreisbewegung mit $r = $ const.
\[\mbox{Zentrifugalkraft } = \mbox{ Lorenzkraft}\]
\[F_Z = F_L\]
\[\frac{m_e v^2}{r} = q_0 \mal v \mal B\]
\textalign{\fbox{$r = \dfrac{m_e}{q_0} \dfrac{v}{B}$}}{$[r] = \frac{kg \mal m \mal m^2}{As \mal s \mal Vs} = \frac{kg \mal m}{s^2} \mal \frac{m^2}{V \mal As}$}

\textalign{$\ra$}{Anwendungen:\\
Ablenkung von $e^-$-Strahlen\\
$\ra$ gr��ere Ablenkwinkel als bei elektro statischer Ablenkung
\begin{itemize}
\item CRT-Monitore
\item \textalign{Synchotron:}{geladene Teilchen auf Kreisbahn ($B$-Feld)\\
		Beschleunigung durch $E$-Feld}
\item REM (B�ndelung von $e^-$-Strahlen und Ablenkung, um Oberfl�che abzurastern)
\end{itemize}}

\subsection{Halleffekt}
22.12.2008-IMG-phys-6
\begin{itemize}
\item Ablenkung von $e^-$ auf eine Seite des Leiters\\
		%arritemize
		\begin{itemize}
		\item $\ra $ $e^-$ �berschuss $\ra$ negativ geladen\\
		\item $\ra$ andere Seite $e^-$ Mangel $\ra$ positiv geladen
		\end{itemize}

		%Arritemize
		\begin{itemize}
		\item Ladungstrennung
		\item elektrische Feld $\vec{e_H}$ (Hallfeld)
		\item Coulombkraft $\vec{F_L}$ wirkt $\vec{F_C}$ entgegen\\
				Gleichgewicht, wenn
				\[F_L =F_C\]
		\end{itemize}
\item 
\item 
\end{itemize}

TODO: Nachtragen Juri 22.12.2008

\section{Neue Section}
\subsection{Lorenzkraft}
$\vec{F_L} senkrechtzeichen \vec{v}$\\
$\vec{F_L} senkrechzeichen \vec{B}$\\
\textalign{$\hookrightarrow$}{
Freie bewegliche Ladungen im Magnetfeld\\
Kreibahn Radius \[r = \frac{m_e}{q_0} \mal \frac{v}{B}\]}
08.01.2009-IMG-phys-1\\
$\vec{F_L} = q \mal \vec{v} \times \vec{B}$\\
08.01.2009-IMG-phys-2\\
$\vec{F_L} = I \mal \vec{l} \times \vec{B}$

\subsection{Halleffekt}
$U_H = - \frac{1}{u \mal q_0} \mal \frac{I \mal B}{d}$
\begin{itemize}[label=$\rightarrow$]
\item Magnetfeldsensor
\item prellfreie Tasten
\end{itemize}
