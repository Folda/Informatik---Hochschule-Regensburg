\chapter{Freie Ladungen}
\textalign{bisher:}{Elektrostatik\\
daher ruhende Ladungen}\\
\textalign{jetzt:}{bewegte Ladungen}\\
Der einfachste Fall der bewegten Ladungen, sind die freien Ladungen im Vakuum.

\section{Erzeugung freier Elektronen}
Elektronen im Metall sind zwar beweglich, aber unterliegen dem Einfluss der positiven Atomr�mpfe\\
$\ra$ Anziehung $\ra$ $e^-$ k�nnen Kristall nicht verlassen.\\
24.11.2008-IMG-phys-5\\
$e^-$ sammeln sich im Potenztopf (Coulombkr�fte durch positive Atomr�mpfe)\\
\textalign{$\ra$}{Energiezufuhr notwendig, um $e^-$ ins Vakuum zu bef�rdern}
\renewcommand{\labelenumi}{\alph{enumi})}
\begin{enumerate}
\item $e^-$ soviel Energie durch Temperatur zuf�hren, dass sie ins Vakuum gelangen k�nnen (''Bindungskr�fte aus Gitter �berwinden'')\\
		27.11.2008-IMG-phys-1

\item Feldemission\\
		Potentialverlauf mit $e$-Feld im Vakuum\\
		\textalign{27.11.2008-IMG-phys-2}{
		$\varphi = - \int \epsilon dx$\\
		$\epsilon =$ const\\
		$\varphi = - \epsilon x$}
		27.11.2008-IMG-phys-3\\
		klassisch k�nnen $e^-$ Potentialwall nicht �berwinden. Jedoch bei Wallbreite im atomaren Ma�stab ($\ra$ hohes $e$-Feld)\\
		\textalign{$\ra$}{quantenmechanischen Tunneleffekt}\\
		\textalign{daher:}{$e^-$ k�nnen Potentialwall durchqueren (''durchtunneln'')}
		\textalign{$\ra$}{Erzeugung eines hohen elektrischen Feldes an Leiteroberfl�che
		\[e \propto \frac{1}{r_{\mbox{oberfl�che}}^2}\]}\\
		\textalign{$\Ra$}{Spritzen mit sehr kleinen Kr�mmungsradien (sehr spitz) $r = 10 \dots 100mmm$\\
		\textalign{$\ra$}{mikrotechnologisch herstellbar}}\\
		\textalign{$\ra$}{FED (Feldemissionsdisplay)\\
		f�r jedes Pixel ein Elektronenstrahl, also eine Spitze\\
		\textalign{Vorteile}{
		\begin{itemize}
		\item d�nn (wie TFT, Plasma)
		\item Farben/Kontrast wie CRT
		\end{itemize}}}
\end{enumerate}
\renewcommand{\labelenumi}{\arabic{enumi}.}

\section{\texorpdfstring{Freie Ladungen im $e$-Feld}{Freie Ladung im elektrischen Feld}}
Auf Ladung $q$ im elektrischen Feld wirkt Coulombkraft
\[F_C = q \mal e\]
$q$ ist frei beweglich $\ra$ Beschleunigung\\
\textalign{$q$}{Verlust bei Bewegung potentieller Energie\\
$\ra$ $q$ gewinnt kinetische Energie(Energieerhaltungsgesetz)}
\[\betrag{\Delta e_{\mbox{pot}} = W_{\mbox{kin}}}\]
27.11.2008-IMG-phys-4\\
Bewegung $A \ra B$\\
\[\Delta e_{\mbox{pot}} = W_{AB} = -q \intx{A}{B}{\vec{e} \mal d\vec{r}}\]
\[\intx{A}{B}{\vec{e} \mal d\vec{r}} = U_{AB}\]
\[\Ra q \mal U_{AB}\]
\[\betrag{\Delta e_{\mbox{pot}}} = \betrag{q U_{AB}} = W_{\mbox{kin}} = \frac{1}{2} m \mal v^2\]
\textalign{$\Ra$}{\fbox{$v = \sqrt{\betrag{2 \frac{2}{m} U_{AB}}}$} (Betrag da $q < 0$, oder $U_{AB} < 0$ m�glich)}\\
\textalign{Beispiel:}{Beschleunigung eines Elektrons\\
Elektron durchl�uft Spannung von $U = 100V$\\
$m_e = \elekmass$\\
$q = - q_0 = - \elemlad$\\
\textalign{Gesucht:}{$v$ nach durchlaufen der Spannungsdifferenz}
\textalign{L�sung:}{
\begin{align*}
	v &= \sqrt{\betrag{2 \frac{-q_0}{m_e} U}}\\
	&= \sqrt{\frac{2 \mal 1,6 \znr{-11} \mal 100}{9,1 \znr{-31}} \frac{As \mal V}{kg}}\\
	V As = J = Nm = \frac{kg \mal mm}{s^2}\\
	v &= 5,9 \znr{6} \frac{m}{s}
\end{align*}
$\ra$ hohe Geschwindigkeit m�glich\\
$\ra$ Gleichung nur g�ltig f�r $v \ll c$ (Lichtgeschwindigkeit)}}

\section{Elektronenstrahlr�hre (Braunsche R�hre)}
\textalign{$\ra$}{CRT, Oszilloskop\\
$\hookrightarrow$ Cathode Ray Tube}\\
Folie $\ra$ Bereiche:
\renewcommand{\labelenumi}{\textcircled{\arabic{enumi}}}
\begin{enumerate}%TODO: Ersten Punkt bei 0 beginnen
\item Emission von $e^-$
		\[v_x = v_y = 0\]

\item Beschleunigung in $x$-Richtung\\
		nach Verlassen des Bereichs:
		\[v_x = \sqrt{\betrag{2 \frac{q_0}{m_e} U_B}}\]

\item ungehinderte Ausbreitung
		\[v_x = \mbox{ const}\]
		\[v_y = \mbox{ const } = 0\]

\item Ablenkung des $e^-$-Strahls in $y$-Richtung
		\[v_x = \mbox{ const}\]
		\begin{alignat*}{2}
			v_y &= a \mal t &&~~~ F = m_e \mal a\\
			&= \frac{F}{m_e} \mal t &&~~~ F = q_0 \mal e = q_0 \mal \frac{U_A}{d}\\
			&= \frac{q_0 \mal U_A}{m_e \mal d} \mal t &&~~~ t = \frac{l}{v_x}
		\end{alignat*}
		\textalign{$\Ra$}{\fbox{$v_y = \dfrac{q_0}{m_e} \mal \dfrac{l}{d} \mal \dfrac{1}{v_x} \mal U_A$}\\
		nach Verlassen des Ablenkkondensators}

\item geradlinige Bewegung mit
		\[v_x = \mbox{ const (siehe oben)}\]
		\[v_y = \mbox{ const (siehe oben)}\]
		Ablenkwinkel $\alpha$
		\begin{align*}
			\tan \alpha &= \frac{v_y}{v_x} = \frac{q_0 \mal l \mal U_A}{m_e \mal d \mal v_x^2}\\
			&= \frac{q_0 \mal l \mal U_A}{m_e \mal d \mal 2 q_0 \mal U_B}
		\end{align*}
		\textalign{$\Ra$}{\fbox{$\tan \alpha = \dfrac{l}{2d} \mal \frac{U_A}{U_B}$}}

\item Auftreffen auf Schirm\\
		($W_{\mbox{kin}}$ der $e^-$ wird in Lichtenergie umgewandelt)\\
		\textalign{$\hookrightarrow$}{Beschleunigung des Schirms wird durch $e^-$-Beschuss zum Leuchten angeregt\\
		$\ra$ Pixel}
		Position auf Schirm:\\
		Dreieck ABC: (Folie 19. unten)
		\[\tan \alpha = \frac{b - s_y}{r}\]
		\[\ra b = s_y + r \tan \alpha\]
		\[v = \frac{d}{dt} s \ra s = \int v dt\]
		\[s_y = \int v_y dt\]
		\begin{align*}
			s_y &= \int v_y dt\\
			&= \int \underbrace{\frac{e_0}{m_e} \frac{U_A}{d}}_{\mbox{const.}} \underbrace{\frac{l}{v_x}}_t dt\\
			&= \frac{e_0}{m_e} \frac{U_A}{d} \mal \frac{1}{2} t^2\\
			&= \frac{e_0}{m_e} \frac{U_A}{d} \mal \frac{1}{2} \mal \frac{l^2}{v_x^2}\\
			&= \frac{e_0}{m_e} \frac{U_A \mal l^2}{2d} \mal \frac{m_e}{2 e_0 U_B}\\
			&= \frac{l}{2} \mal \frac{U_A \mal l}{U_B \mal 2d} = \frac{l}{2} \tan \alpha = s_y 
		\end{align*}
		$\ra$ ''wie Spiegel bei $\frac{l}{2}$''\\
		\[\begin{array}{lll}
			\Ra &b &= \frac{l}{2} \tan \alpha + r \tan \alpha\\
			&&= \tan \alpha \mal \left(\frac{l}{2} + r\right)
		\end{array}\]
		\fbox{$b = \left(\dfrac{l}{2} + r\right) \mal \dfrac{l}{2d \mal U_B} U_A$}
		$\Ra b \propto U_A$
\end{enumerate}
\renewcommand{\labelenumi}{\arabic{enumi}.}
