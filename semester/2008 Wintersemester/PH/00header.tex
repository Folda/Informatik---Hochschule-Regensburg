%Hefter
\documentclass[fontsize=10pt,twoside=false,a4paper,fleqn,parskip=half]{scrreprt}
%neue Rechtschreibung
\usepackage{ngerman}
%Umlaute erm�glichen
\usepackage[latin1]{inputenc}
%Fontkodierung
\usepackage[T1]{fontenc}
\newcommand{\changefont}[3]{\fontfamily{#1} \fontseries{#2} \fontshape{#3} \selectfont}
%Einstellungen der Seitenr�nder
\usepackage[left=2.5cm,right=2.5cm,top=2.5cm,bottom=2cm,includeheadfoot]{geometry}
%Erweitertes Unterstreichen
\usepackage{ulem}
%Umrahmen
\usepackage{fancybox}
%Mathematische Pakete und Fonts
\usepackage{amsmath}
\usepackage{amsfonts}
\usepackage{polynom} %Polynomdivison Darstellen
\usepackage{mathrsfs}
\usepackage{wasysym}
%Verschiedene Symbole
\usepackage{amssymb}
\usepackage{latexsym}
\usepackage{marvosym}
%Bilder
\usepackage{graphicx}
%Tabellen
\usepackage{array}
%Links
\usepackage{hyperref}
%Inhaltsverzeichnis
\usepackage{index}
%Farben
\usepackage[usenames]{color}

\usepackage{float}

\usepackage{longtable}
\usepackage{libertine}

\usepackage{textcomp}
\usepackage{gensymb}

\usepackage{varioref}
\usepackage{enumitem}
%Diagramme
\usepackage{tikz}

%Quellcode Einf�gen
\usepackage{listings} \lstset{numbers=left, numberstyle=\small, numbersep=5pt}
\lstset{
language=bash,
stringstyle=\ttfamily,
showstringspaces=false
}

%%%%%%%%Commandos%%%%%%%%%%%%%%%%%%%%%%%%%%%%%%%%%%%%%%%%%%%%
%%%%%%%%Entspricht
\newcommand{\equals}{\stackrel{\scriptscriptstyle\wedge}{=}}

%%%%%%%%Zehnerpotenzen
\newcommand{\znr}[1]{\cdot 10^{#1}}

%%%%%%%%Betrag
\newcommand{\betrag}[1]{\left| #1 \right|}

%%%%%%%%Sin, Cos, Tan
\newcommand{\sinx}[1]{\sin{\left( #1 \right)}} %%Sin
\newcommand{\cosx}[1]{\cos{\left( #1 \right)}} %%Cos
\newcommand{\tanx}[1]{\tan{\left( #1 \right)}} %%Tan

%%%%%%%%Arabische in R�mische Zahl umwandeln
\newcommand{\RM}[1]{\MakeUppercase{\romannumeral #1}}

%%%%%%%%Langer Vektor
\newcommand{\lvec}[1]{\overrightarrow{#1}}

%%%%%%%%Ausgeschriebener Vektor
\newcommand{\vektor}[3]{\begin{pmatrix} #1\\#2\\#3 \end{pmatrix}}

%%%%%%%%Ausgeschriebener Punkt
\newcommand{\punkt}[4]{#1 \left( \begin{array}{c|c|c} #2 & #3 & #4 \end{array} \right)}

%%%%%%%%Eingesetzt in
\newcommand{\tin}{\mbox{ in }}

%%%%%%%%In Anf�hrungszeichen Setzen
\newcommand{\quotate}[1]{\glqq #1\grqq }

%%%%%%%%In geschweifte Klammern setzen
\newcommand{\gklamm}[1]{$\left\{ \mbox{#1} \right\}$}

%%%%%%%%Kreis um Text Zeichnen
\newcommand{\textkreis}[1]{\unitlength1ex\begin{picture}(2.5,2.5)%
\put(0.75,0.75){\circle{2.5}}\put(0.75,0.75)(\makebox(0,0){#1}\end{picture}}

%%%%%%%%Begrenztes Int
\newcommand{\intx}[3]{\ensuremath{\int_{#1}^{#2} #3}}

%%%%%%%TextAlign und FakeTextAlign
\newcommand{\textalign}[2]{
\begin{minipage}[b]{\widthof{#1} + \widthof{\space}}
#1
\end{minipage}
\begin{minipage}[t]{\linewidth-\widthof{#1}-\widthof{\space}}
%\vspace{-6pt}#2
#2
\end{minipage}
}

\newcommand{\textfakealign}[2]{
\begin{minipage}[b]{\widthof{#1} + \widthof{\space}}
$ $
\end{minipage}
\begin{minipage}[t]{\linewidth-\widthof{#1}-\widthof{\space}}
%\vspace{-6pt}#2
#2
\end{minipage}
}

%%%%%%%Sum und Prod
\newcommand{\sumx}[3]{\ensuremath{\sum_{#1}^{#2} #3}}
\newcommand{\prodx}[3]{\ensuremath{\prod_{#1}^{#2} #3}}

% \if\blank --- checks if parameter is blank (Spaces count as blank) 
% \if\given --- checks if parameter is not blank: like \if\blank{#1}\else 
% \if\nil --- checks if parameter is null (spaces are NOT null) 
% use \if\given{ } ... \else ... \fi etc. 
% Beispiel: \newcommand{\blah}[1]{\if\blank{#1}Leer\else#1\fi}
% 
{\catcode`\!=8 % funny catcode so ! will be a delimiter 
\catcode`\Q=3 % funny catcode so Q will be a delimiter 
\long\gdef\given#1{88\fi\Ifbl@nk#1QQQ\empty!} 
\long\gdef\blank#1{88\fi\Ifbl@nk#1QQ..!}% if null or spaces 
\long\gdef\nil#1{\IfN@Ught#1* {#1}!}% if null 
\long\gdef\IfN@Ught#1 #2!{\blank{#2}} 
\long\gdef\Ifbl@nk#1#2Q#3!{\ifx#3}% same as above 
}

%%%%%%%%Verschiedene Konstanten
%%%%%%%%Elektrische Feldkonstante
\def \elefeldk { 8,854 \cdot 10^{-12} \frac{F}{m} }
%%%%%%%%Gravitationskonstante
\def \gravik { 6,673 \cdot 10^{-11} \frac{m^3}{kg s^2} }
%%%%%%%%Elementarladung
\def \elemlad { 1,602 \cdot 10^{-19} C }
%%%%%%%%Elektronenmasse
\def \elekmass { 9,109 \cdot 10^{-31} kg }
%%%%%%%%Protonenmasse
\def \protomass { 1,673 \cdot 10^{-27} kg }

%%%%%%%%Abk�rzungen
\newcommand{\Ra}{\ensuremath{\Rightarrow}}
\newcommand{\ra}{\ensuremath{\rightarrow}}
\newcommand{\Lra}{\ensuremath{\Leftrightarrow}}
\newcommand{\mal}{\ensuremath{\cdot}}
\newcommand{\relmeng}{\ensuremath{\mathbb{R}}}
\newcommand{\ganzmeng}{\ensuremath{\mathbb{Z}}}
\newcommand{\natmeng}{\ensuremath{\mathbb{N}}}
\newcommand{\und}{\ensuremath{\wedge}}
\newcommand{\oder}{\ensuremath{\vee}}
\newcommand{\aeq}{\ensuremath{\Leftrightarrow}}
\newcommand{\ohne}{\ensuremath{\backslash}}
%<>%%%%%Commandos%%%%%%%%%%%%%%%%%%%%%%%%%%%%%%%%%%%%%%%%%%%%
