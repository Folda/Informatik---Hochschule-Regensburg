\chapter{Invertierbare Matrizen}
\section{Definitionen}
\subsection{Quadratische Matrix, regul�re Matrix}
\label{sec:MatrizenundDeterminanten-4-7}
Man nennt eine Matrix $A$ quadratisch\index{quadratisch}, wenn sie gleich viele Spalten wie Zeilen hat, also $A \in M(n, n)$ f�r ein $n \in \mathbb{N}$.\\
Eine quadratische Matrix $A$ nennt man regul�r\index{regul�r}, falls die Spalten von $A$ linear unabh�ngig sind.\\
\textalign{Bemerkung:}{$A$ regul�r $A \in M(n, n)$\\
$\Leftrightarrow \func{f_A} \mathbb{R}^n \ra \mathbb{R}^n$, bijektiv\\
$\Leftrightarrow \func{f_A} \mathbb{R}^n \ra \mathbb{R}^n$, surjektiv\\
$\Leftrightarrow \func{f_A} \mathbb{R}^n \ra \mathbb{R}^n$, injektiv}

\section{S�tze}
\subsection{Existenz der Inversen Matrix}
\label{sec:MatrizenundDeterminanten-4-8}
F�r jede quadratische Matrix $A \in M(n, n)$ sind die folgenden Aussagen �quivalent:
\begin{enumerate}
\item $A$ ist regul�r
\item es gibt ein $B \in M(n, n)$ mit $AB = E_n$
\item es gibt ein $B \in M(n, n)$ mit $BA = E_n$
\item es gibt ein $B \in M(n, n)$ mit $AB = E_n = BA$
\end{enumerate}
Ist eine dieser Aussagen erf�llt, ist $B$ regul�r und durch eine der Gleichungen (1. oder 2.) eindeutig bestimmt. Man nennt $B$ dann die Inverse\index{Inverse} von $A$, Zudem bezeichnet man $A$ dann auch als invertierbar\index{invertierbar}.\\
\textalign{Bemerkung:}{$A$ regul�r $\Leftrightarrow A$ ist invertierbar.}\\
\textalign{Beweis:}{Sei $\func{f_A} \mathbb{R}^n \ra \mathbb{R}^n$ sei die zu $A$ geh�rende lineare Abbildung.\\
\textalign{$i \Ra$ 2., 3. ,4.}{da $A$ regul�r ist $f_A$ bijektiv Daher existiert die Umkehrabbildung $f_A^{-1} \mathbb{R}^n \ra \mathbb{R}^n$ ($f_A^{-1}$ ist eindeutig), so dass man nach Satz \vref{sec:Grundlagen-1-24} gilt:
\[f_A^{-1} \circ f_A = id_{\mathbb{R}^n}\]
\[f_A \circ f_A^{-1} = id_{\mathbb{R}^n}\]
Ist $B$ dann die Standardmatrix von $f_A^{-1}$ gilt:
\[B \mal A = E_n\]
\[A \mal B = E_n\]}

\textalign{3. $\Ra$ 1.}{es existiert eine Matrix $B$ mit $BA = E_n$\\
Wenn $f_B$ die zu $B$ geh�rende lineare Abbildung ist gilt also
\[f_B \circ f_A = id_{\mathbb{R}^n}\]
$id_{\mathbb{R}^n}$ ist bijektiv $\Ra$ (mit Aufgabe E 13 a,c) $f_B$ ist surjektiv, $f_A$ ist injektiv. Da wir lineare Abbildungen $\mathbb{R}^n \ra \mathbb{R}^n$ betrachten gilt $f_A, f_B$ bijektiv $\Ra$ A,B regul�r.}

\textalign{2. $\Ra$ 1.}{\demonstrand{analog zu 3. $\Ra$ 1.}}}

\textalign{Bemerkung:}{F�r die Inverse von $A$ schreibt man $A^{-1}$ (in Anlehnung an $f^{-1}$)}

\subsection{Weiterer Satz}
\label{sec:MatrizenundDeterminanten-4-9}
Gegeben sei eine regul�re Matrix $A \in M(n, n)$ und ein Vektor $\vec{b} \in M(n, 1)$. Dann ist $A^{-1} \mal \vec{b}$ die eindeutige L�sung des (quadratischen) linearen Gleichungssystems.
\[A \vec{x} = \vec{b}\]
\textalign{Beweis:}{$A \in M(n ,n)$ regul�r $\Leftrightarrow \func{f_A} \mathbb{R}^n \ra \mathbb{R}^n$ bijektiv daher wegen Surjektivit�t ist jedes $\vec{b} \in \mathbb{R}^n$ Bild eines Punktes aus $\mathbb{R}^n$ unter $f_A$, daher das Gleichungssystem $A \vec{x} = \vec{b} \left(\Leftrightarrow f_A (\vec{x}) = \vec{b}\right)$ hat mindestens eine L�sung\\
wegen Injektivit�t (jeder Punkt kann h�chstens das Bild eines Punktes sein) ist diese L�sung eindeutig.\\
Bleibt also zz dass diese L�sung die Form
\[\vec{x} = A^{-1} \mal \vec{b} \mbox{ hat}\]
\demonstrand{\[A \vec{x} = A \mal \left(A^{-1} \vec{b}\right) = \left(A \mal A^{-1}\right) \mal \vec{b} = En \vec{b} = \vec{b}\]}}

Nachtragen von Jurij - Donnerstag 04.12.2008

\section{Ungeordnet}
\subsection{1.2.1}
$\gauss{ccc|c}{1 & 0 & 1 & 1\\0 & \lambda & 1 & 0\\1 & 1 & 0 & 0}$
$\underrightarrow{\RM{3} = \RM{3} - \RM{1}} \gauss{ccc|c}{1 & 0 & 1 & 1\\0 & \lambda & 1 & 0\\0 & 1 & 1 & 1}$
$\underrightarrow{\RM{2} \leftrightarrow \RM{3}} \gauss{ccc|c}{1 & 0 & 1 & 1\\0 & 1 & 1 & 1\\0 & \lambda & 1 & 0}$
$\underrightarrow{\RM{3} = \RM{3} - \lambda \RM{2}} \gauss{ccc|c}{1 & 0 & 1 & 1\\0 & 1 & 1 & 1\\0 & 0 & 1 - \lambda & -\lambda}$

\textalign{Fall 1:}{$1 - \lambda = 0 \Leftrightarrow \lambda = 1$\\
$\gauss{ccc|c}{1 & 0 & 1 & 1\\0 & 1 & 1 & 1\\0 & 0 & 0 & 1}$\\
$\Ra L = \emptyset$}

\textalign{Fall 2:}{$\lambda = 0$\\
$\gauss{ccc|c}{1 & 0 & 1 & 1\\0 & 1 & 1 & 1\\0 & 0 & 1 & 0}$
$\underrightarrow{\RM{2} = \RM{2} - \RM{3}} \gauss{ccc|c}{1 & 0 & 1 & 1\\0 & 1 & 0 & 1\\0 & 0 & 1 & 0}$
$\underrightarrow{\RM{1} = \RM{1} - \RM{3}} \gauss{ccc|c}{1 & 0 & 0 & 1\\0 & 1 & 0 & 1\\0 & 0 & 1 & 0}$\\
$\Ra L = \gklamm{(1, 1, 0)^T}$\\
$\Ra$ Konsistent f�r $\lambda = 0$}

\subsection{1.2.2}
$p(x) = a_0 + a_1 x + a_2 x^2$\\
$p(1) = 12$, $p(2) = 15$, $p(3) = 16$
08.12.2008-IMG-mathe-1\\
$a_0 + 1 \mal a_1 + 1 \mal a_2 = 12$\\
$a_0 + 2 \mal a_2 + 2^2 a_2 = 15$\\
$a_0 + 3 \mal a_1 + 3^2 \mal a_2 = 16$\\
$\underrightarrow{\RM{2} = \RM{2} - \RM{1}}$\\
$\underrightarrow{\RM{3} = \RM{3} - \RM{1}}$\\
$\underrightarrow{\RM{3} = \RM{3} - 2 \RM{2}}$\\
$\underrightarrow{\RM{3} = \RM{3} \frac{1}{2}}$\\
$\underrightarrow{\RM{2} = \RM{2} - 3 \mal \RM{3}}$\\
$\underrightarrow{\RM{1} = \RM{1} - \RM{3}}$\\
$\underrightarrow{\RM{1} = \RM{1} - \RM{2}}$\\
$\Ra a_0 = 7, a_1 = 6, a_2 = -1$\\
$\Ra$ Polynom $p(x) = 7 + 6x -x^2$

\subsection{� 2.5}
$\gauss{ccc|c}{1 & 0 & 5 & 2\\-2 & 1 & -6 & -1\\0 & 2 & 8 & x}$
$\underrightarrow{\RM{2} = \RM{2} + 2 \RM{1}}$\\
$\underrightarrow{\RM{3} = \RM{3} - 2 \RM{2}}$

\textalign{Fall 1:}{$x - 6 \neq 0$\\
letzte Spalte ist eine Pivot-Spalte $\Ra$ es existiert keine L�sung $\Ra$ $\vec{b}$ ist keine Linearkombination der Spalten von $A$}

\textalign{Fall 2:}{$x - 6 = 0 \Leftrightarrow x = 6$\\
$\gauss{ccc|c}{1 & 0 & 5 & 2\\0 & 1 & 4 & 3\\0 & 0 & 0 & 0}$\\
es existiert L�sung $\Ra$ $\vec{b}$ ist Linearkombination $a$ Spalten von $A$\\
$\Ra x = 6$}

\subsection{� 2.6}
\renewcommand{\labelenumi}{\alph{enumi})}
\begin{enumerate}
\item $\vec{u_1} = \vektor{1}{-5}{2}, \vec{u_2} = \vektor{-3}{8}{6}, \vec{u_3} = \vektor{4}{\alpha}{-8}$\\
		$\gauss{ccc|c}{1 & -3 & 4 & 0\\-5 & 8 & \alpha & 0\\-2 & 6 & -8 & 0}$
		$\underrightarrow{\RM{2} = \RM{2} + 2 \RM{1}} \gauss{ccc|c}{1 & -3 & 4 & 0\\-5 & 8 & \alpha & 0\\0 & 0 & 0 & 0}$\\
		da homogenes Gleichungssystem existiert mindesten eine L�sung da zudem eine freie Variable vorhanden ist existieren $\infty$-viele L�sungen.
		\renewcommand{\labelitemi}{$\Rightarrow$}
		\begin{itemize}
		\item Die Vektoren sind f�r alle $\alpha$ Linear abh�ngig.
		\item Es existiert kein $\alpha \in \mathbb{R}$ so, dass $\vec{u_1}, \vec{u_2}, \vec{u_3}$ linear unabh�ngig sind.
		\end{itemize}
		\renewcommand{\labelitemi}{$\bullet$}

\item $\vec{v_1} = \varvektor{c}{1\\-5}, \vec{v_2} = \varvektor{c}{1\\7}, \vec{3} = \varvektor{c}{-2\\\alpha}$\\
		da im 2-dimensionalen Raum maximal 2 Vektoren unabh�ngig sein k�nnen, existiert kein $\alpha$ so, dass $\vec{v_1}, \vec{v_2}, \vec{v_3}$ linear unabh�ngig sind.
\end{enumerate}
\renewcommand{\labelenumi}{\arabic{enumi}.}
