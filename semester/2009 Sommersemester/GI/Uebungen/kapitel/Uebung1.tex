\chapter{1. �bung}
\section{Aufgaben}
\subsection{1. Aufgabe (2+2+2+2+2 Punkte)}
Gegeben sei das ALphabet $\Sigma = \gklamm{0, 1}$. Ferner seien $L_1 = \gklamm{0, 00, 000}$ und $L_1 = \gklamm{0, 1, 01}$. F�r welche Sprachen stehen die folgenden Ausdr�cke?
\begin{enumerate}[label=(\alph*)]
\item $L_1 \mal L_2$
\item $L_1^3$
\item $L_1^C$
\item $L_1^*$
\item $\emptyset^*$
\end{enumerate}

\subsection{2. Aufgabe (3+3 Punkte)}
Geben Sie induktive Definitionen f�r die folgenden Funktionen an:
\begin{enumerate}[label=(\alph*)]
\item $\del$: $\Sigma^* \times \Sigma \ra \Sigma^*$ entfernt alle Vorkommen eines Symbols aus einem Wort (z.B. $\del(abbcab, a) = bbcb$).
\item $\gicount$: $\Sigma^* \times \Sigma \ra \mb{N}$ z�hlt alle Vorkommen eines Symbols in einem Wort (z.B. $\gicount$ ist $count(abbdd, d) = 2$).
\end{enumerate}

\subsection{3. Aufgabe (3+3+3 Punkte)}
Gegeben sei das Alphabet $\Sigma = \gklamm{0, 1, \sharp}$.
\begin{enumerate}[label=(\alph*)]
\item Wie viele verschiedene W�rter der L�nge $n$gibt es?
\item Wie viele verschiedene W�rter der L�nge $n$ gibt es, in denen ein Symbol $c \in \Sigma$ genau $k$-mal vorkommt?
\item Wie viele verschiedene W�rter der L�nge $n$ gibt es, in denen ein Symbol $c \in \Sigma$ mindestens $k$-mal vorkommt?
\end{enumerate}

\subsection{4. Aufgabe (10 Zusatzpunkte)}
Gegeben sei das Alphabet $\Sigma_{\tx{Bool}} = \gklamm{0, 1}$. Geben Sie zwei Beispielsprachen $L_1$ und $L_2$ an, so dass die folgende Aussage gilt:
\[(L_1 \cup L_2)^* = L_1^* \cup L_2^*\]
Dabei d�rfen $L_1$ und $L_2$ weder gleich, noch leer sein und m�ssen sich von $\gklamm{0, 1}^*$ unterscheiden. Beweisen Sie ihre Aussage.

\section{Eigene L�sung}
\subsection{1. Aufgabe}
$\Sigma = \gklamm{0, 1}$, $L_1 = \gklamm{0, 00, 000}$, $L_2 = \gklamm{0, 1, 01}$
\begin{enumerate}[label=(\alph*)]
\item $L_1 \mal L_2 = \gklamm{00, 01, 001, 000, 0001, 0000, 00001}$
\item $L_1^3 = \gklamm{0^3, 0^4, 0^5, 0^6, 0^7, 0^8, 0^9}$
\item $L_1^{\mathcal{C}} = \Sigma^* - L_1$
\item $L_1^* = \gklamm{\epsilon, 0, 00, 000, 0^4, \dots}$
\item $\emptyset^* = \gklamm{\epsilon}$
\end{enumerate}

\subsection{2. Aufgabe}
\begin{enumerate}[label=(\alph*)]
\item $\del$
		\begin{itemize}
		\item $\del(\epsilon) = \epsilon$
		\item sei $w = av$, falls $a = b$ wobei $b$ der zu l�schende Buchstabe ist, sei $\del(w) = del(v)$. Ansonsten ist $\del(w) = a \del(v)$.
		\end{itemize}

\item $\gicount$
		\begin{itemize}
		\item $\gicount(\epsilon) = 0$
		\item sei $w = av$, falls $a = b$ wobei $b$ der zu z�hlende Buchstabe sei ist $\gicount(w) = \gicount(v) + 1$. Ansonsten ist $\gicount(w) = \gicount(v) + 0$.
		\end{itemize}
\end{enumerate}

\subsection{3. Aufgabe}
$\Sigma = \gklamm{0, 1, \sharp}$
\begin{enumerate}[label=(\alph*)]
\item Die Anzahl der verschiedenen W�rter ist die Anzahl der Zeichen in $\Sigma$ hoch $n$ wobei $n$ die Anzahl der Stellen ist. In dem Fall $\Sigma = \gklamm{0, 1, \sharp}$ ist die Anzahl also $n^3$.
\item Es gibt $\matrixp{n\\k}$ M�glichkeiten das Symbol zu platzieren. F�r jede Platzierung hat man eine M�glichkeit weniger ein also $(n - k)$. Daraus ergibt sich die Formel:
		\[\matrixp{n\\k} \mal 2^{(n - k)}\]

\item Die Antwort ergibt sich aus der vorherigen L�sung nur muss man alle einzelnen Wahrscheinlichkeiten des Zeichens aufsummieren. Es folgt also:
		\[\sum_{i = k}^n \matrixp{n\\i} \mal 2^{n - i}\]
\end{enumerate}

\subsection{4. Aufgabe}
$\Sigma_{\text{Bool}} = \gklamm{0, 1}$, $\underbrace{\rklamm{L_1 \cup L_2}^*}_{M_1} 0 \underbrace{L_1^* \cup L_2^*}_{M_2}$
\subsection*{Beispielsprachen}
$L_1 = \gklamm{01}$\\
$L_2 = \gklamm{10}$

\subsubsection{Beweis der Aussage}
\subsubsection*{Annahme}
$M_1 \subseteq M_2 \wedge M_2 \subseteq M_1$\\
\Ra $L_1^* \subseteq \rklamm{L_1 \cup L_2}^* \wedge L_2^* \subseteq \rklamm{L_1 \cup L_2}^*$

\subsubsection{Beweis}
\begin{enumerate}
\item $M_1 \subseteq M_2$
		\begin{align*}
		u &\in \rklamm{L_1 \cup L_2}^*\\
		u &\in \rklamm{\gklamm{01} \cup \gklamm{10}}^*\\
		u &= \gklamm{01}\\
		&\Ra u \in \rklamm{\gklamm{01} \cup \gklamm{10}}^* \text{ da } \gklamm{01} \in \gklamm{01}^*
		\end{align*}
		\Ra $u \in M_1$
\item $M_2 \subseteq M_1$
		\begin{align*}
		u &\in L_1^* \cup L_2^*\\
		u &\in \gklamm{01}^* \cup \gklamm{10}^*\\
		u &= \gklamm{01}\\
		&\Ra u \in \gklamm{01}^* \cup \gklamm{10}^* \text{ da } \gklamm{01} \in \gklamm{01}^*
		\end{align*}
		\Ra $u \in M_1$
\end{enumerate}
\Ra $u \in M_1 \wedge M_2$ \Ra $M_1 = M_2$
\begin{flushright}$\blacksquare$\end{flushright}

\section{Muster L�sung}
\subsection{1. Aufgabe}
$L_1 = \gklamm{0,00,0000}$, $L_2 = \gklamm{0,1,0}$,
		\begin{enumerate}[label=\alph*)]
		\item $L_1 \mal L_2 = \gklamm{0,00,000} \mal \gklamm{0, 1, 01} = \gklamm{00,01,000,001,0000,0001,00001}$
		\item $L_1^3 = L_1 \mal L_1 \mal L_1 = \gklamm{0^2,0^3,0^4,0^5,0^6} \mal \gklamm{0,0^2,0^5} = \gklamm{0^3, 0^4, 0^5, 0^6, 0^7, 0^8, 0^9}$
		\item $L_1^\mathcal{C} = \Sigma^* - L_1 = \gklamm{\epsilon, 1, 01, 10, 11,001,010,011,100,101, \dots} = \Sigma^* - \gklamm{0,00,0000}$
		\item $L_1^* = \gklamm{\epsilon,0,00,000,0000,\dots} = \gklamm{0}^*$
		\item $\emptyset^* = \gklamm{\epsilon}$
		\end{enumerate}

\subsection{2. Aufgabe}
\begin{enumerate}[label=(\alph*)]
		\item \begin{enumerate}[label=\arabic*.)]
				\item $\del(\epsilon, a) = \epsilon$
				\item $\del(w \mal a, b) = \begin{cases}\del(w, b) \tx{ falls } a = b\\\del(w, b) \mal a \tx{ sonst}\end{cases}$
				\end{enumerate}

		\item \begin{enumerate}[label=\arabic*.)]
				\item $\gicount(\epsilon, a) = 0$
				\item $\gicount(w \mal a, b) = \begin{cases}\gicount (w, b) \tx{ falls } a \neq b\\\gicount(w, b) + 1 \tx{ sonst}\end{cases}$
				\end{enumerate}
		\end{enumerate}

\subsection{3. Aufgabe}
$\Sigma = \gklamm{0, 1, \sharp}$
\begin{enumerate}[label=(\alph*)]
\item $\underbrace{3 \mal 3 \mal \dots \mal 3}_{n} = 3^n$
\item $\underbrace{\binom{n}{k}}_{} \mal 2^{(n - k)}$
\item $\sum_{i = k}^{n} \binom{n}{l} \mal 2^{(n - i)}$
\end{enumerate}

\subsection{4. Aufgabe}
$(L_1 \cup L_2)^* = L_1^* \cup L_2^*$\\
$L_1 \neq L_2; L_1, L_2 \neq \{\}; L_1, L_2 \neq \gklamm{0,1}^*$\\
$L_1$ so dass
\[L_1^* = (L_1 \cup L_2)^*\]
\[(L_1 \cup L_2)^* = (L_1 \cup L_2)^* \cup L_2^*\]
$L_1 = \gklamm{0, 01}$, $L_2 = \gklamm{0, 00}$\\
$L_0 = \gklamm{\epsilon} \cup \gklamm{0w \vert \tx{ zwischen je zwei } \wedge \tx{ ist in } w \tx{ mindestens eine } 0}$\\
$L_1^* = (L_1 \cup L_2)^* = L_0$\\
$L_1^* \subseteq (L_1 \cup L_2)^* \subseteq L_0 \subseteq L_1^*$\\
zu $(L_1 \cup L_2)^* \subseteq L_0$: $v \in (L_1 \cup L_2)^*$\\
$v = a_1, a_2 \dots a_k (1 \klgl i \klgl k)$ f�r $k \in \mb{N}$, $a_i \in L_1 \cup L_2 = \gklamm{0, 01, 00}$\\
$u = a_2 \dots a_k$
\Ra $v \in L_0$\\
zu $L_0 \subseteq L_1^*$: $v \in L_0$
\begin{itemize}
\item $v \in \gklamm{0}^* \subseteq L_1^*$
\item $ = v_1 01 v_2 01 v_3 01 \dots v_{k - 1} 01 v_k$\\
		$v_i \in \gklamm{0}^* (1 \klgl i \klgl k)$\\
		$\gklamm{0}^* \subseteq L_1^*$ und $\gklamm{01} \in L_1^* \ra v \in L_1^*$\\
		$L_1 = \gklamm{0, 01}$, $L_2 = \gklamm{00}$
\item 
\end{itemize}
