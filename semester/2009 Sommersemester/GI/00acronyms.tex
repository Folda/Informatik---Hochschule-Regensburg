\begin{acronym}
	\acro{z.B.}[\ensuremath{\mbox{z.\,B.}\xspace}]{zum Beispiel}
	\acro{bel.}{beliebigem}
	\acro{Def.}{Definition}
	\acro{gdw.}[\ensuremath{\mbox{g.\,d.\,w.}\xspace}]{genau dann wenn}
	\acro{def.}{definiert}
	\acro{Opt.}{Optimiert}
	\acro{DEA}{Deterministischer endlicher Automat}
	\acro{akzept.}{akzeptierende}
	\acro{bzw.}{beziehungsweise}
	\acro{bzgl.}{bez�glich}
	\acro{NEA}{Nichtdeterministische endliche Automaten}
	\acro{Ber.}{Berechnung}
	\acro{Berechn.}{Berechnung}
	\acro{Bew.}{Beweis}
	\acro{d.h.}[\ensuremath{\mbox{d.\,h.}\xspace}]{daher}
	\acro{akz.}{akzeptierende}
	\acro{ex.}{existiert}
	\acro{Anw.}{Anwendung}
	\acro{M�gl.}{M�glichkeit}
	\acro{mind.}{mindestens}
	\acro{incl.}{inklusiv}
	\acro{Bsp.}{Beispiel}
	\acro{ex.}{existiert}
	\acro{PMK}{Potenzmengenkonstruktion}
	\acro{vollst.}{vollst�ndige}
	\acro{indukt.}{induktive}
	\acro{IV}{Induktionsvoraussetzung}
	\acro{abgeschl.}{abgeschlossen}
	\acro{PL}[\ensuremath{\mbox{P.\,L.}\xspace}]{Pumping Lemma}
	\acro{RS}{Regul�re Sprache}
	\acro{gem.}{gem��}
	\acro{bekanntem.}{bekannterma�en}
	\acro{reg.}{regul�rer}
	\acro{RA}{Regul�re Ausdruck}
	\acro{G-NEA}{generalisierter nicht deterministischer Automat}
	\acro{WS}{Wertsprache}
	\acro{bel.}{beliebige}
	\acro{o.B.d.A}[\ensuremath{\mbox{o.\,B.\,d.\,A}\xspace}]{ohne Beschr�nkung der Allgemeinheit}
	\acro{Terminalsymb.}{Terminalsymbole}
	\acro{s.o.}[\ensuremath{\mbox{s.\,o.}\xspace}]{siehe oben}
	\acro{TM}{Turing Maschine}
	\acro{S-/L-Kopf}{Schreib-/Lesekopf}
\end{acronym}
