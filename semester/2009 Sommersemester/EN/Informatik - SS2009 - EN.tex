\documentclass[oneside,a4paper,parskip=half]{scrbook}
\usepackage[latin1]{inputenc}

%Seitenheader
\usepackage{fancyhdr}
\pagestyle{fancy}

%Standartfont
\usepackage[T1]{fontenc}
\usepackage{libertine}

%Zusatzpaket f�r mathematische Ausdr�cke
\usepackage{amsmath}

%Zusatzfonts f�r mathbb usw.
\usepackage{amsfonts}
\usepackage{mathrsfs}

%Verbessertes Ref
\usepackage{varioref}

%Links
\usepackage{hyperref}

%Stichwortverzeichnis
\usepackage{makeidx}
\makeindex

%Acronyme
\usepackage[nolist,nohyperlinks]{acronym}

%Anpassbare Enumerates/Itemizes
\usepackage{enumitem}

%Paket zum Berechnen von Textbreiten und H�hen
\usepackage{calc}

%F�r das manipulieren von Captions in Figuren
\usepackage{caption}

%Um die Seite in mehrere Spalten aufzuteilen
\usepackage{multicol}

%Tikz/PGF Zeichnenpaket
\usepackage{tikz}

% \if\blank --- checks if parameter is blank (Spaces count as blank) 
% \if\given --- checks if parameter is not blank: like \if\blank{#1}\else 
% \if\nil --- checks if parameter is null (spaces are NOT null) 
% use \if\given{ } ... \else ... \fi etc. 
% Beispiel: \newcommand{\blah}[1]{\if\blank{#1}Leer\else#1\fi}
% 
{\catcode`\!=8 % funny catcode so ! will be a delimiter 
\catcode`\Q=3 % funny catcode so Q will be a delimiter 
\long\gdef\given#1{88\fi\Ifbl@nk#1QQQ\empty!} 
\long\gdef\blank#1{88\fi\Ifbl@nk#1QQ..!}% if null or spaces 
\long\gdef\nil#1{\IfN@Ught#1* {#1}!}% if null 
\long\gdef\IfN@Ught#1 #2!{\blank{#2}} 
\long\gdef\Ifbl@nk#1#2Q#3!{\ifx#3}% same as above 
}

%<Commandos>
%In geschweifte Klammern setzen
\newcommand{\gklamm}[1]{\ensuremath{\left\{#1\right\}}}

%In eckige Klammern setzen
\newcommand{\eklamm}[1]{\ensuremath{\left\[#1\right\]}}

%In Betragsstriche Setzen
\newcommand{\betrag}[1]{\ensuremath{\left|#1\right|}}

%Script zum Eingeben von l�ngeren Beispielen
\newcommand{\bsp}[3][]
{
\if\blank{#3}\textalign{\textbf{#2}}{#1}\else
\textbf{#2} #1
\par
\begingroup
\leftskip=1.28em
\setlist[1]{labelindent=1.28em, leftmargin=*}
#3
\par
\endgroup\fi
}

%Script um Text einzur�cken
\newcommand{\textalign}[2]{
\begin{minipage}[b]{\widthof{#1} + \widthof{\space}}
#1
\end{minipage}
\begin{minipage}[t]{\linewidth-\widthof{#1}-\widthof{\space}}
#2
\end{minipage}
}

%Script um Text einzur�cken ohne Ausgabe
\newcommand{\textfakealign}[3]{
\begin{minipage}[b]{\widthof{#1} + \widthof{\space}}
\if\blank{#2}$ $\else#2\fi
\end{minipage}
\begin{minipage}[t]{\linewidth-\widthof{#1}-\widthof{\space}}
#3
\end{minipage}
}

%Indexe
%Normaler Index
\newcommand{\indexn}[2][]{#2\if\blank{#1}\index{#2}\else\index{#1}\fi}
%Unterstrichener Index
\newcommand{\indexu}[2][]{\underline{#2}\if\blank{#1}\index{#2}\else\index{#1}\fi}
%Kursiver Index
\newcommand{\indexi}[2][]{\textit{#2}\if\blank{#1}\index{#2}\else\index{#1}\fi}
%Fetter Index
\newcommand{\indexb}[2][]{\textbf{#2}\if\blank{#1}\index{#2}\else\index{#1}\fi}

%%%%%%%%Arabische in R�mische Zahl umwandeln
\newcommand{\RM}[1]{\ensuremath{\mbox{\MakeUppercase{\romannumeral #1}}}}
%</Commandos>

%<Abk�rzungen>
\newcommand{\Ra}{\ensuremath{\Rightarrow}}
\newcommand{\ra}{\ensuremath{\rightarrow}}
\newcommand{\Lra}{\ensuremath{\Leftrightarrow}}
\newcommand{\lra}{\ensuremath{\leftrightarrow}}
\newcommand{\hra}{\ensuremath{\hookrightarrow}}
\newcommand{\mul}{\ensuremath{\cdot}}

%Sin, Cos, Tan, Dim, Span, Arccos
\DeclareMathOperator{\spano}{span}

\newcommand{\sinx}[1]{\ensuremath{\sin{\left(#1\right)}}}
\newcommand{\cosx}[1]{\ensuremath{\cos{\left(#1\right)}}}
\newcommand{\tanx}[1]{\ensuremath{\tan{\left(#1\right)}}}
\newcommand{\dimx}[1]{\ensuremath{\dim{\left(#1\right)}}}
\newcommand{\spanx}[1]{\ensuremath{\spano{\left(#1\right)}}}
\newcommand{\arccosx}[1]{\ensuremath{\arccos{\left(#1\right)}}}

%Schriften
\newcommand{\mb}[1]{\ensuremath{\mathbb{#1}}}

%Misc
\newcommand{\mal}{\ensuremath{\cdot}}
%</Abk�rzungen>




%<Einstellung f�r Hyperref>%%%%%%%%%%%%%%%%%%%%%%%%%%%%%%%%%%%%%%%%%%%%%%%%%%%%
\hypersetup{colorlinks=false, linkcolor=black, breaklinks=true, bookmarksdepth=3,unicode=true,bookmarksnumbered=true,
pdftitle={Hefter f�r Englisch -- Stand: \today},
pdfauthor={Thaller Alexander},
pdfsubject={Hefter f�r das Fach Englisch 2 (EN2) aus der Vorlesung von Professor Christopher Inman f�r das 2. Semester der Informatik im Sommersemester 2009 an der Hochschule Regensburg.},
pdfkeywords={informatik,studium,hefter,englisch,sommersemester,2009}}
%</Einstellung f�r Hyperref>%%%%%%%%%%%%%%%%%%%%%%%%%%%%%%%%%%%%%%%%%%%%%%%%%%%

%<Dokument Daten>%%%%%%%%%%%%%%%%%%%%%%%%%%%%%%%%%%%%%%%%%%%%%%%%%%%%%%%%%%%%%%
\titlehead{
\begin{minipage}[t]{5cm}
Alexander Thaller\\
Marktstra�e 22\\
92331 Lupburg
\end{minipage}
\begin{minipage}[t]{34mm}
\includegraphics[width=34mm]{FH-Logo.pdf}
\end{minipage}
}
\subject{Informatik (Bachelor) 2. Semester}
\author{Alexander Thaller}
\title{Englisch 2\footnote{Gefundene Fehler oder Verbesserungsvorschlge bitte hier \href{http://fhrein2ss09.codeplex.com/WorkItem/List.aspx}{http://fhrein2ss09.codeplex.com/WorkItem/List.aspx} berichten oder alternativ mir eine E-Mail schreiben \href{mailto:alexander.thaller@stud.fh-regensburg.de}{alexander.thaller@stud.fh-regensburg.de}. Vielen Dank.}}
\subtitle{Hefter des Sommersemesters 2009}
\date{stand \today}
\publishers{Aus der Vorlesung von Professor Christopher Inman}
%</Dokument Daten>%%%%%%%%%%%%%%%%%%%%%%%%%%%%%%%%%%%%%%%%%%%%%%%%%%%%%%%%%%%%%

\begin{document}
\begin{acronym}
	\acro{acr}{Akronym}
	\acro{z.B.}{zum Beispiel}
	\acro{bzw.}{Beziehungsweise}
	\acro{d.h.}{daher}
	\acro{Abb.}{Abbildung}
\end{acronym}



\maketitle
\setcounter{tocdepth}{1}
\setcounter{secnumdepth}{2}
\tableofcontents
\newpage
%<Text>%%%%%%%%%%%%%%%%%%%%%%%%%%%%%%%%%%%%%%%%%%%%%%%%%%%%%%%%%%%%%%%%%%%%%%%%
%\chapter{Intel chips get power boost}
\textit{Every fifth word of the first two paragraphs has been shortened. Complete the words so that the text makes good sense. Each underscore (\_) represents one missing letter.}

Intel has released updated v\ugap{item:Page1_FillGaps_G1}{7} of its Pentium 4 c\ugap{item:Page1_FillGaps_G2}{4} that help desktop machines ha\ugap{item:Page1_FillGaps_G3}{4} their growing role as v\ugap{item:Page1_FillGaps_G4}{4}, sound and image-editing s\ugap{item:Page1_FillGaps_G5}{6}. The Prescott series of P\ugap{item:Page1_FillGaps_G6}{6} 4 chips are tuned to \gap{2}(\ref{item:Page1_FillGaps_G7}) a better job of h\ugap{item:Page1_FillGaps_G8}{7} multimedia as well as a r\ugap{item:Page1_FillGaps_G9}{4} of scientific and engineering a\ugap{item:Page1_FillGaps_G10}{11}. Intel has also given \gap{3}(\ref{item:Page1_FillGaps_G11}) chip more memory and m\ugap{item:Page1_FillGaps_G12}{3} changes to the way \gap{2}(\ref{item:Page1_FillGaps_G13}) handles instructions in an at\ugap{item:Page1_FillGaps_G14}{5} to improve performance. The t\ugap{item:Page1_FillGaps_G15}{9} giant has also modified its m\ugap{item:Page1_FillGaps_G16}{12} methods for the new t\ugap{item:Page1_FillGaps_G17}{3} of chip to help i\ugap{item:Page1_FillGaps_G18}{1} take performance to new l\ugap{item:Page1_FillGaps_G19}{5}.

\section{Solutions}
\begin{multicols}{2}
\begin{enumerate}
\item version \label{item:Page1_FillGaps_G1}
\item chips \label{item:Page1_FillGaps_G2}
\item handle \label{item:Page1_FillGaps_G3}
\item video \label{item:Page1_FillGaps_G4}
\item systems \label{item:Page1_FillGaps_G5}
\item Pentium \label{item:Page1_FillGaps_G6}
\item do \label{item:Page1_FillGaps_G7}
\item handling \label{item:Page1_FillGaps_G8}
\item range \label{item:Page1_FillGaps_G9}
\item applications \label{item:Page1_FillGaps_G10}
\item the \label{item:Page1_FillGaps_G11}
\item made \label{item:Page1_FillGaps_G12}
\item it \label{item:Page1_FillGaps_G13}
\item attempt \label{item:Page1_FillGaps_G14}
\item technology \label{item:Page1_FillGaps_G15}
\item manufacturing \label{item:Page1_FillGaps_G16}
\item type \label{item:Page1_FillGaps_G17}
\item it \label{item:Page1_FillGaps_G18}
\item levels \label{item:Page1_FillGaps_G19}
\end{enumerate}
\end{multicols}

\section{Industrial action}
\begin{multicols}{2}
\begin{enumerate}
\item to
\item the
\item immediately
\item had
\item stressing
\item chip
\item make
\item conductor
\item new
\item the
\item current
\item ensure
\item smoothly
\end{enumerate}
\end{multicols}

\begin{multicols}{2}
\begin{enumerate}
\item giant
\item modify
\item leak
\item cram
\item alteration
\item supersede
\item collectively
\item review
\item steadily
\item accelerate
\end{enumerate}
\end{multicols}
\begin{enumerate}
\item very big and dominant (A very large company with a great market share / which is dominant in it's sector)
\item to change an existing thing (to make changes to part of a process, system \ac{etc.} without changing the whole process)
\item to slowly loose things from an containment (unwanted/unintended emission from a closed system)
\item to insert the same amount of something into a smaller place with very few room (to put/push/force more and more items into a small space)
\item an alternative version of something or doing something with some few changes
\item to make something better or faster than previous versions (replace the old version)\\
		\Ra The USB quickly superseded the parallel and RS232 ports
\item to target a group of things (together/as a group)
\item to watch and use something and rate it after specific criteria
\item to not change something over a period of time
\item to make something faster than before (increase)
\end{enumerate}

\subsection{Questions}
\begin{enumerate}
\item Because the old chips, which were used in desktops, weren't able to handle the amount of data which is used in multimedia applications. (Because image editing and multimedia applications need greater CPU\\$\begin{cases}\text{speed}\\\text{performance}\end{cases}$.)
\item The silicon becomes a better conductor. (The silicon becomes a better conductor\\$\begin{cases}\text{which makes the chip more efficient}\\\text{which increases processing speed}\end{cases}$.)
\item The loss of power. (This can lead to loss of data, processing errors, loss of performance generally and a shorter processor lifespan.)
\item The process is much larger and so has more space for memory. (The decreased width of components from 130nm to 90nm.)
\item There is older software which isn't using the advantage of the new instructions of the processor. (Because current applications have not been designed for the new chips (are not yet compatible with\dots).)
\item It allows the use of new processors with the old chip sets/motherboards. (Intel can produce the new chips without changing the production line, which saves a large amount of money.)
\end{enumerate}


%\section{20.04.2009}
%\begin{enumerate}
%\item higher capacity (The higher capacity. - The times greater capacity.)
%\item 51\%
%\item They can securely disposed with simple tools (Standard DVDs cannot be disposed of easily, and sensitive data can be retrieved from discs which people throw away.)
%\item In the near future (THis is not known. - This has not been announced yet.)
%\item They use a blue laser instead of a red one (Paper discs will use blue laser technology, where current conventional DVDs use red laser.)
%\item 25GB
%\item 4.7GB
%\item Because paper discs cannot hold an substrate which is needed by the red laser technology. (Red laser light needs to pass through a substrate to read and write data. Paper cannot easily bond with a substrate. Red laser light would set fire to the paper.)
%\item They are cheaper in production because of the missing substrate and the cheapness of paper. (They can be produced more cheaply.)
%\item They they say that they don't know what the practical uses could be. (They say they don't know.)
%\end{enumerate}

%\chapter{Hard drive speed limit is reached}
%\section{Vocabulary}
%\section{Questions}
%\begin{enumerate}
%\item It is not very practical.\\
%		Because the writing speed in practice depends in many more details/factors such as the speed of the movement of the positioning arm.\\
%		Because present computers are far away from working with electrons shot at the speed of light.
%
%\item They flip a bit faster than current materials.\\
%		Because the properties of exotic materials could possibly allow higher speeds.
%
%\item Changing the magnetic charge.\\
%		To change the bit to the opposite state.\\
%		The state of a bit changes from 0 to 1 or vice versa.
%
%\item So they do not blank a hole part of the disk.\\
%		Because a long-lived pulse would affect neighbouring bits.\\
%		So that the bits are not blocked from further change.
%
%\item To accelerate materials.\\
%		Accelerate charged/subatomic particles to a high speed.
%
%\item The effect is to small to affect the material.\\
%		There is an effect but it is not reliable enough to use this in hard-disk production.
%\end{enumerate}
%\section{Sentence structure}
%\begin{enumerate}
%\item Ka
%\item Ka
%\item Ka
%\item Ka
%\item Ke
%\end{enumerate}

\begin{enumerate}
\item The runtime is too short.\\
		The runtime of laptop batteries is too short.\\
		The limited runtime of the batteries.

\item NEC\\
		A Japanese concern called NEC.

\item Using fuel cells which can run up to 5 hours.\\
		A fuel cell. Its performance is five hours.\\
		A fuel cell that has/with a runtime of 5 hours.

\item methanol, several ounces\\
		300cc of methanol.

\item They use a catalyst to separate electrons and protons which leads to an electrical charge.\\
		By a catalytic reaction which yields heat and power.\\
		A catalyst assists the breakdown of methanol into oxygen and hydrogen which generates/yields heat and power.

\item It is heavier than normal batteries.\\
		The weight of the laptop increases by 2kg.

\item They will run up to 40 hours.\\
		A runtime for the fuel cell of about 40 hours.

\item They are cramping more batteries into the devices.\\
		They are trying to reduce the power needed/consumed by hardware.\\
		They have build more power efficient components for computers.

\item They believe they have already reached the limit of power efficiency.
\item Mobile phones/Cell Phones.
\end{enumerate}

\begin{enumerate}
\item \textbf{The access time is} the time between the receipt of instructions from the processor and the onset of data transfer by the programmer.
\item \textbf{The output buffers} are the data interface between memory and processor.\\
		\textbf{The output buffers} are a part of memory which temporarily stores data until it can be processed.

\item \textbf{A wait state is} a condition in which the processor can do nothing/must remain idle until another component completes an operation.
\end{enumerate}

\begin{enumerate}
\item particles
\item circuits
\item regardless
\item fundamental
\item implies
\item ultimate
\item absurd
\item supplanted
\item bound
\item observed
\end{enumerate}

\begin{enumerate}
\item Distance doesn't affect the relationship.
\item No
\item Because all his laws state that nothing can travel faster than the speed of light.
\item Because it didn't fit in any other law of that time.
\item The reaction speed of the materials which are used in the computer.
\item The usage of light instead of electrons.
\item Because it implies that there is no longer a speed limit.
\end{enumerate}
%</Text>%%%%%%%%%%%%%%%%%%%%%%%%%%%%%%%%%%%%%%%%%%%%%%%%%%%%%%%%%%%%%%%%%%%%%%%

%Literaturverzeichnis
\newpage
\addcontentsline{toc}{part}{Literaturverzeichnis}
\bibliography{literatur}{}

%Stichwortverzeichnis
\newpage
\renewcommand{\indexname}{Stichwortverzeichnis}
\addcontentsline{toc}{part}{Stichwortverzeichnis}
\printindex

\end{document}


