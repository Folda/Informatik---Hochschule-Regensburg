%Theoreme
\theoremstyle{definition}
\newtheorem{definition}{Definition}[chapter]
\newtheorem{example}{Beispiel}[chapter]
\newtheorem{exercise}{Aufgabe}[chapter]

\theoremstyle{remark}
\newtheorem*{note}{Bemerkung}
\newtheorem{notation}{Notation}[chapter]

\theoremstyle{plain}
\newtheorem{theorem}{Satz}[chapter]
\newtheorem{lemma}[theorem]{Lemma}


%Umgebungen
\newenvironment{automat}[1][2.8cm]
{\begin{tikzpicture}[>=stealth',auto,node distance=#1,semithick,initial text=]}
{\end{tikzpicture}} %Tikz Picture - Automaten

%Klammern
\newcommand{\gklamm}[1]{\ensuremath{\left\{#1\right\}}}	%Geschweifte Klammern
\newcommand{\rklamm}[1]{\ensuremath{\left(#1\right)}}	%Runde Klammern
\newcommand{\betrag}[1]{\ensuremath{\left|#1\right|}}	%Betragsstriche

%Mathematische Zeichen
\newcommand{\coloneqq}{\ensuremath{\mathrel{\mathop:\!\!=}}}		%Zeichen für "ist definiert durch die linke seite"
\newcommand{\eqqcolon}{\ensuremath{\mathrel{=\!\!\mathop:}}}		%Zeichen für "ist definiert durch die rechte seite"
\newcommand{\potenzmenge}[1]{\ensuremath{\mathcal{P}\rklamm{#1}}}		%Funktion der Potenzmenge

%Griechische Buchstaben
\newcommand{\Alpha}{\ensuremath{\mbox{A}}}
\newcommand{\Beta}{\ensuremath{\mbox{B}}}
\newcommand{\Iota}{\ensuremath{\mbox{I}}}
\newcommand{\Rho}{\ensuremath{\mbox{R}}}
\newcommand{\Kappa}{\ensuremath{\mbox{K}}}
\newcommand{\Tau}{\ensuremath{\mbox{T}}}
\newcommand{\Mu}{\ensuremath{\mbox{M}}}
\newcommand{\Epsilon}{\ensuremath{\mbox{E}}}
\newcommand{\Nu}{\ensuremath{\mbox{N}}}
\newcommand{\Zeta}{\ensuremath{\mbox{Z}}}
\newcommand{\Chi}{\ensuremath{\mbox{X}}}
\newcommand{\Eta}{\ensuremath{\mbox{H}}}
\newcommand{\Omicron}{\ensuremath{\mbox{O}}}
\newcommand{\omicron}{\ensuremath{\mbox{o}}}

%Abkürzungen
\newcommand{\ceq}{\ensuremath{\coloneqq}}		%Abkürzung zu coloneqq
\newcommand{\eqc}{\ensuremath{\eqqcolon}}		%Abkürzung zu eqqcolon
\newcommand{\gk}[1]{\ensuremath{\gklamm{#1}}}		%Abkürzung für gklamm
\newcommand{\rk}[1]{\ensuremath{\rklamm{#1}}}		%Abkürzung für rklamm
\newcommand{\tx}[1]{\ensuremath{\text{#1}}}		%Abkürzung für text
\newcommand{\enq}[1]{\enquote{#1}}			%Abkürzung für enquote
\newcommand{\oder}{\ensuremath{\vee}}			%Abkürzung für vee (oder Zeichen)
\newcommand{\und}{\ensuremath{\wedge}}			%Abkürzung für wedge (und Zeichen)
\newcommand{\bigoder}{\ensuremath{\bigvee}}			%Abkürzung für bigvee (oder Zeichen)
\newcommand{\bigund}{\ensuremath{\bigwedge}}			%Abkürzung für bigwedge (und Zeichen)
\newcommand{\Ra}{\ensuremath{\Rightarrow}}		%Abkürzung für Rightarrow
\newcommand{\ra}{\ensuremath{\rightarrow}}		%Abkürzung für rightarrow
\newcommand{\La}{\ensuremath{\Leftarrow}}		%Abkürzung für Leftarrow
\newcommand{\la}{\ensuremath{\leftarrow}}		%Abkürzung für leftarrow
\newcommand{\Lra}{\ensuremath{\Leftrightarrow}}		%Abkürzung für Leftrightarrow
\newcommand{\lra}{\ensuremath{\leftrightarrow}}		%Abkürzung für leftrightarrow
\newcommand{\klgl}{\ensuremath{\leq}}			%Abkürzung für leq (kleiner gleich)
\newcommand{\grgl}{\ensuremath{\geq}}			%Abkürzung für geq (größer gleich)
\newcommand{\ul}[1]{\underline{#1}}			%Abkürzung für underline
\newcommand{\ob}[2]{\ensuremath{\overbrace{#1}^{#2}}}	%Abkürzung für overbrace
\newcommand{\ub}[2]{\ensuremath{\underbrace{#1}_{#2}}}	%Abkürzung für overbrace
\newcommand{\btg}[1]{\ensuremath{\betrag{#1}}}		%Abkürzung für betrag
\newcommand{\mal}{\ensuremath{\cdot}}			%Abkürzunf für cdot
\newcommand{\ol}[1]{\ensuremath{\overline{#1}}}		%Abkürzung für overline
\newcommand{\N}{\ensuremath{\mathbb{N}}}			%Abkürzung für die Menge der Natürlichen Zahlen
\newcommand{\Z}{\ensuremath{\mathbb{Z}}}			%Abkürzung für die Menge der Ganzen Zahlen
\newcommand{\Q}{\ensuremath{\mathbb{Q}}}			%Abkürzung für die Menge der Rationalen Zahlen
\newcommand{\R}{\ensuremath{\mathbb{R}}}			%Abkürzung für die Menge der Reellen Zahlen
\newcommand{\pmeng}[1]{\ensuremath{\potenzmenge{#1}}}	%Abkürzung für die Funktion der Potzenzmenge
