%Provides UTF8 encoding
\usepackage[utf8x]{inputenc}

%Sets the language of the document to german
%\usepackage[ngerman]{babel}
\usepackage[ngerman]{babel}

%Use german quotes
\usepackage[babel=true]{csquotes}

%For better usage of the typorgaphic features of pdftex
\usepackage[tracking=true,kerning=true,spacing=true,final,babel]{microtype}

%For support of inserting graphics (jpeg, png, ...)
\usepackage{graphicx}

%Makes links avaivable in the pdf output
\usepackage[pdftex]{hyperref}
\usepackage[all]{hypcap} %Fixes the position a link points to see http://en.wikibooks.org/wiki/LaTeX/Hyperlinks#Problems_with_tables_and_figures

%For mathematical symbols
\usepackage{amsmath}

%For mathematical fonts like mathbb
\usepackage{amsfonts}

%For customizable theoremes
\usepackage{amsthm}

%For better looking tables
\usepackage{booktabs}

%Acronyme
\usepackage[nolist,nohyperlinks]{acronym}

%For special spaces between characters
\usepackage{xspace}

%Better ref
\usepackage[ngerman]{varioref}

%For TikzPictures
\usepackage{pgf}
\usepackage{tikz}

%For multiple columns in the text
\usepackage{multicol}

%For sourcecode
\usepackage{listings}
