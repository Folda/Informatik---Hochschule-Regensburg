\section{�bung 2}

\begin{tikzpicture}[node distance = 2.5cm, auto]
	\node[cloud] (start){Start};
	\node[block, right of=start] (eingabe1){$JB$};
	\node[block, right of=eingabe1] (eingabe2){$N = 1$};
	\node[block, right of=eingabe2] (eingabe3){$I = J$};
	\node[block, below of=eingabe3] (berechnung1){$K_N = I \% B$\\$I = I \div B$};
	\node[block, below of=berechnung1] (berechnung2){$N = N + 1$};
	\node[decision, below of=berechnung2] (decision1){$I \neq 0$};
	\node[block, below of=decision1] (berechnung3){$N = N -1$};
	\node[block, below of=berechnung3] (ausgabe1){Ausgabe $K_N$};
	\node[decision, below of=ausgabe1] (decision2){$N > 1$};
	\node[cloud, below of=decision2] (ende){Ende};

	\path [line] (start.east) -- (eingabe1.west);
	\path [line] (eingabe1.east) -- (eingabe2.west);
	\path [line] (eingabe2.east) -- (eingabe3.west);
	\path [line] (eingabe3.south) -- (berechnung1.north);
	\path [line] (berechnung1.south) -- (berechnung2.north);
	\path [line] (berechnung2.south) -- (decision1.north);
	\path [line] (decision1.east) -| node[near start]{Ja} ++(2,0) |- (berechnung1.east);
	\path [line] (decision1.south) -- node[near start]{Nein} (berechnung3.north);
	\path [line] (berechnung3.south) -- (ausgabe1.north);
	\path [line] (ausgabe1.south) -- (decision2.north);
	\path [line] (decision2.west) -| node[near start]{Ja} ++(-2,0) |- (berechnung3.west);
	\path [line] (decision2.south) -- node[near start]{Nein} (ende.north);
\end{tikzpicture}
\subsection{L�sung zu Aufgabe 2}
\subsubsection{L�sungsbeschreibung}
Genutzt wird die iterative Euklitische Methode mit Division die im Pseudocode wie folgt lautet:
\begin{lstlisting}[language=pseud]
EUCLID(a,b)
   solange b != 0
       h <- a mod b
       a <- b
       b <- h
   return a
\end{lstlisting}
(Siehe \url{http://de.wikipedia.org/wiki/Euklidischer_Algorithmus#Iterative_Variante})\\
In C w�rde der Pseudocode wie folgt aussehen:
\begin{lstlisting}[language=c]
#include <stdio.h>

int main()
{
    int wert1, wert2, a, b, h;
    
    printf("Erste Zahl eingeben\n");
    scanf("%d", &wert1);
    printf("Zweite Zahl eingeben\n");
    scanf("%d", &wert2);
    a = wert1;
    b = wert2;
    
    printf("Starte die Berechnung...\n");
    while(b != 0)
    {
         h = a % b;
         a = b;
         b = h;
    }
    printf("Die Berechnung ist beendet...\n\n");
    
    printf("%d und %d haben %d als kleinsten gemeinsammen Teiler\n",
            wert1, wert2, a);
    
    system("pause");
    return 0;
}
\end{lstlisting}
\subsubsection{L�sung}    
\begin{tikzpicture}[node distance = 4cm, auto]
	\node[cloud] (start){Start};
	\node[block, right of=start] (setab) {$a = b$ und $b = h$};
	\node[decision, below of=setab] (decide) {$b_n = 0$};
	\node[block, below of=decide] (ausgabe) {Ausgabe von $a$};
	\node[block, left of=decide] (eingabe){Eingabe $a \neq 0$ und $b$};
	\node[block, right of=decide] (berechnung) {$h = a \% b$};

	\path [line] (start.south) -- (eingabe.north);
	\path [line] (eingabe.east) -- (decide.west);
	\path [line] (decide.south) -- node[near start]{Ja} (ausgabe.north);
	\path [line] (decide.east) -- node[near start]{Nein} (berechnung.west);
	\path [line] (berechnung.north) |- (setab.east);
	\path [line] (setab.south) -- (decide.north);
\end{tikzpicture}
\subsection{L�sung zu Aufgabe 3}
\renewcommand{\labelenumi}{\alph{enumi})}
\renewcommand{\labelenumii}{\arabic{enumii}.}
\begin{enumerate}
\item Das Steuerwerk (englisch Control Unit, kurz CU) ist der Kern des Mikroprozessors. Seine Aufgabe ist die Abarbeitung des Programms, d.h. Befehl f�r Befehl eines Programms werden durch das Steuerwerk des Mikroprozessors ausgef�hrt.\\\\
		Es bestehte aus:
		\begin{itemize}
		\item Befehlsz�hler
		\item Befehlsregister
		\item Decodiereinheit
		\item Ablaufsteuerung
		\end{itemize}
		(Quelle: \url{http://de.wikipedia.org/wiki/Steuerwerk})
\item \begin{enumerate}
		\item Dekodierung des Befehls (\lstinline!DECODE!)
		\item Laden der Operanden die auf den Befehl angewandt werden sollen (\lstinline!FETCH OPERANDS!)
		\item Ausf�hrung des Befehls (\lstinline!EXECUTE!)
		\item Befehlsz�hler aktualisieren (\lstinline!UPDATE INSTRUCTION POINTER!)
		\end{enumerate}
		(Quelle: \url{http://de.wikipedia.org/wiki/Steuerwerk})
\item Bei bedingten Sprungbefehlen wird zun�chst die logische Bedingung (beispielsweise ein Test, ob der Akkumulatorinhalt gr��er, gleich oder kleiner null ist) aufgewertet. Falls die Bedingung erf�llt ist, wird der Befehlsz�hler mit einer ermittelten oder explizit angegebenen Sprungadresse aus dem Restanteil des Befehlsregisters in den Befehlsz�hler �bertragen.
\end{enumerate}
\renewcommand{\labelenumi}{\arabic{enumi}.}
\renewcommand{\labelenumii}{\alph{enumii})}
