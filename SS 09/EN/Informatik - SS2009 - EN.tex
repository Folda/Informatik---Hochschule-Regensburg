\documentclass[oneside,a4paper,german,parskip=half,draft]{scrbook}

\usepackage[latin1]{inputenc}

%Neue Deutsche Rechtschreibung
\usepackage{ngerman}

%Seitenheader
\usepackage{fancyhdr}
\pagestyle{fancy}
\rhead{\nouppercase{\rightmark}}
\lhead{\nouppercase{\leftmark}}

%Standartfont
\usepackage[T1]{fontenc}
%\usepackage{fourier}
%\usepackage{libertine}

%Zusatzpaket f�r mathematische Ausdr�cke
\usepackage{amsmath}

%Zusatzfonts f�r mathbb usw.
\usepackage{amsfonts}
\usepackage{mathrsfs}

%Verbessertes Ref
\usepackage[german]{varioref}

%Links
\usepackage{hyperref}

%Stichwortverzeichnis
\usepackage{makeidx}
\makeindex

%Acronyme
\usepackage[nolist,nohyperlinks]{acronym}

%Anpassbare Enumerates/Itemizes
\usepackage{enumitem}

%Tikz/PGF Zeichnenpaket
\usepackage{pgf}
\usepackage{tikz}
\usetikzlibrary{mindmap,trees,decorations,decorations.pathreplacing,decorations.pathmorphing,calc,arrows,automata}

%F�r graphische Todos
\usepackage[colorinlistoftodos, obeyDraft]{todonotes}

%Paket zum Berechnen von Textbreiten und H�hen
\usepackage{calc}

%BibTeX
%\usepackage{cite}
%\usepackage{bibgerm}
%\bibliographystyle{gerplain}

%F�r das manipulieren von Captions in Figuren
\usepackage{caption}

%F�r das anpassen von theoremen
\usepackage{amsthm}

%F�r die farbigen Boxen
\usepackage{xcolor}
\usepackage{framed}

%Um die Seite in mehrere Spalten aufzuteilen
\usepackage{multicol}

%F�r zwischenr�ume in W�rtern
\usepackage{xspace}

%Quellcode Einbindung
\usepackage{listings}
\lstset{
basicstyle=\ttfamily
}

%F�r das durchstreichen von Bl�cken
\usepackage{cancel}

%F�r Commandos mit zwei opotionalen Argumenten
\usepackage{twoopt}

% \if\blank --- checks if parameter is blank (Spaces count as blank) 
% \if\given --- checks if parameter is not blank: like \if\blank{#1}\else 
% \if\nil --- checks if parameter is null (spaces are NOT null) 
% use \if\given{ } ... \else ... \fi etc. 
% Beispiel: \newcommand{\blah}[1]{\if\blank{#1}Leer\else#1\fi}
% 
{\catcode`\!=8 % funny catcode so ! will be a delimiter 
\catcode`\Q=3 % funny catcode so Q will be a delimiter 
\long\gdef\given#1{88\fi\Ifbl@nk#1QQQ\empty!} 
\long\gdef\blank#1{88\fi\Ifbl@nk#1QQ..!}% if null or spaces 
\long\gdef\nil#1{\IfN@Ught#1* {#1}!}% if null 
\long\gdef\IfN@Ught#1 #2!{\blank{#2}} 
\long\gdef\Ifbl@nk#1#2Q#3!{\ifx#3}% same as above 
}

%<Commandos>
%In geschweifte Klammern setzen
\newcommand{\gklamm}[1]{\ensuremath{\left\{#1\right\}}}

%In eckige Klammern setzen
\newcommand{\eklamm}[1]{\ensuremath{\left[#1\right]}}

%In runde Klammern setzen
\newcommand{\rklamm}[1]{\ensuremath{\left(#1\right)}}

%In Betragsstriche Setzen
\newcommand{\betrag}[1]{\ensuremath{\left|#1\right|}}

%Script zum Eingeben von l�ngeren Beispielen
\newcommand{\bsp}[3][]
{
\if\blank{#3}\textalign{\textbf{#2}}{#1}\else
\textbf{#2} #1
\par
\begingroup
\leftskip=1.28em
\setlist[1]{labelindent=1.28em, leftmargin=*}
#3
\par
\endgroup\fi
}

%Script um Text einzur�cken
\newcommand{\textalign}[2]{
\begin{minipage}[b]{\widthof{#1} + \widthof{\space}}
#1
\end{minipage}
\begin{minipage}[t]{\linewidth-\widthof{#1}-\widthof{\space}}
#2
\end{minipage}
}

%Script um Text einzur�cken ohne Ausgabe
\newcommand{\textfakealign}[3]{
\begin{minipage}[b]{\widthof{#1} + \widthof{\space}}
\if\blank{#2}$ $\else#2\fi
\end{minipage}
\begin{minipage}[t]{\linewidth-\widthof{#1}-\widthof{\space}}
#3
\end{minipage}
}

%Indexe
%Normaler Index
\newcommand{\indexn}[2][]{#2\if\blank{#1}\index{#2}\else\index{#1}\fi}
%Unterstrichener Index
\newcommand{\indexu}[2][]{\underline{#2}\if\blank{#1}\index{#2}\else\index{#1}\fi}
%Kursiver Index
\newcommand{\indexi}[2][]{\textit{#2}\if\blank{#1}\index{#2}\else\index{#1}\fi}
%Fetter Index
\newcommand{\indexb}[2][]{\textbf{#2}\if\blank{#1}\index{#2}\else\index{#1}\fi}

%Definitionen, Beispiele, S�tze
\newenvironment{fshaded}[2]{%
\def\FrameCommand{\fcolorbox{#1}{#2}}%
\MakeFramed {\FrameRestore}}%
{\endMakeFramed}

\theoremstyle{definition}
\newtheorem{definition}{\ac{Def.}}[part]
\newenvironment{fdefinition}[1][]{\definecolor{shadecolor_definition}{rgb}{.87,.94,.96}
\definecolor{framecolor_definition}{rgb}{0,0,0}%
\begin{fshaded}{framecolor_definition}{shadecolor_definition}\begin{definition}[#1]}{\end{definition}\end{fshaded}}

\newtheorem{satz}[definition]{Satz}
\newenvironment{fsatz}[1][]{\definecolor{shadecolor_satz}{rgb}{.87,.96,.56}
\definecolor{framecolor_satz}{rgb}{0,0,0}%
\begin{fshaded}{framecolor_satz}{shadecolor_satz}\begin{satz}[#1]}{\end{satz}\end{fshaded}}

\newtheorem{beweis}{Beweis}[part]
\newenvironment{fbeweis}[1][]{\definecolor{shadecolor_beweis}{rgb}{1.00,.96,.92}
\definecolor{framecolor_beweis}{rgb}{0,0,0}%
\begin{fshaded}{framecolor_beweis}{shadecolor_beweis}\begin{beweis}[#1]}{\end{beweis}\end{fshaded}}

\theoremstyle{remark}
\newtheorem{notation}{Notation}[part]
\newenvironment{fnotation}[1][]{\definecolor{shadecolor_notation}{rgb}{.90, .90, .90}
\definecolor{framecolor_notation}{rgb}{0,0,0}%
\begin{fshaded}{framecolor_notation}{shadecolor_notation}\begin{notation}[#1]}{\end{notation}\end{fshaded}}

\newtheoremstyle{note}% name
	{1em}			%Space above
	{1em}			%Space below
	{}				%Body font
	{}				%Indent amount (empty = no indent, \parindent = para indent)
	{\bfseries}	%Thm head font
	{:}			%Punctuation after thm head
	{.5em}		%Space after thm head: " " = normal interword space; \newline = linebreak
	{}				%Thm head spec (can be left empty, meaning `normal')

\newtheoremstyle{notelite}% name
	{3pt}			%Space above
	{3pt}			%Space below
	{}				%Body font
	{}				%Indent amount (empty = no indent, \parindent = para indent)
	{\itshape}	%Thm head font
	{:}			%Punctuation after thm head
	{.5em}		%Space after thm head: " " = normal interword space; \newline = linebreak
	{}				%Thm head spec (can be left empty, meaning `normal')

\theoremstyle{note}
\newtheorem*{beispiel}{Beispiel}

\theoremstyle{notelite}
\newtheorem*{anmerkung}{Anmerkung}
\newtheorem*{beobachtung}{Beobachtung}
\newtheorem*{bemerkung}{Bemerkung}
\newtheorem*{achtung}{Achtung}
%%%%%%%%Arabische in R�mische Zahl umwandeln
\newcommand{\RM}[1]{\ensuremath{\mbox{\MakeUppercase{\romannumeral #1}}}}
\newcommand{\rM}[1]{\ensuremath{\mbox{\romannumeral #1}}}
%</Commandos>

%<Abk�rzungen>
\newcommand{\Ra}{\ensuremath{\Rightarrow}}
\newcommand{\ra}{\ensuremath{\rightarrow}}
\newcommand{\La}{\ensuremath{\Leftarrow}}
\newcommand{\la}{\ensuremath{\leftarrow}}
\newcommand{\Ral}{\ensuremath{\Longrightarrow}}
\newcommand{\ral}{\ensuremath{\longrightarrow}}
\newcommand{\Lra}{\ensuremath{\Leftrightarrow}}
\newcommand{\lra}{\ensuremath{\leftrightarrow}}
\newcommand{\hra}{\ensuremath{\hookrightarrow}}
\newcommand{\mul}{\ensuremath{\cdot}}
\newcommand{\grgl}{\ensuremath{\geq}}
\newcommand{\klgl}{\ensuremath{\leq}}
\newcommand{\oder}{\ensuremath{\vee}}
\newcommand{\und}{\ensuremath{\wedge}}
\newcommand{\mal}{\ensuremath{\cdot}}
\newcommand{\matrixp}[1]{\ensuremath{\begin{pmatrix} #1 \end{pmatrix}}}
\newcommand{\coloneqq}{\mathrel{\mathop:\!\!=}}
\newcommand{\ceq}{\ensuremath{\coloneqq}}
\newcommand{\tx}[1]{\ensuremath{\text{#1}}}
\newcommand{\rkl}[1]{\ensuremath{\rklamm{#1}}}
\newcommand{\N}{\ensuremath{\mb{N}}}
\newcommand{\R}{\ensuremath{\mb{R}}}
\newcommand{\Z}{\ensuremath{\mb{Z}}}
\newcommand{\stack}[2]{\ensuremath{\stackrel{#1}{#2}}}
\newcommand{\ub}[2]{\ensuremath{\underbrace{#1}_{#2}}}

%Todo, Insert, Mark, Img
\newcommand{\Todo}[1]{\todo[inline]{TODO: #1}}
\newcommand{\Insert}[1]{\todo[inline, color=green!40]{INSERT: #1}}
\newcommand{\Mark}[1]{\todo[inline, nolist, color=blue!40]{MARK: #1}}
\newcommand{\Img}[1]{\todo[inline, nolist, color=black!40]{IMG: #1}}

%Sin, Cos, Tan, Dim, Span, Arccos, Limes
\DeclareMathOperator{\spano}{span}
\DeclareMathOperator{\lb}{lb}

\newcommand{\sinx}[1]{\ensuremath{\sin{\left(#1\right)}}}
\newcommand{\cosx}[1]{\ensuremath{\cos{\left(#1\right)}}}
\newcommand{\tanx}[1]{\ensuremath{\tan{\left(#1\right)}}}
\newcommand{\dimx}[1]{\ensuremath{\dim{\left(#1\right)}}}
\newcommand{\spanx}[1]{\ensuremath{\spano{\left(#1\right)}}}
\newcommand{\arccosx}[1]{\ensuremath{\arccos{\left(#1\right)}}}

%Schriften
\newcommand{\mb}[1]{\ensuremath{\mathbb{#1}}}
%</Abk�rzungen>




%<Einstellung f�r Hyperref>%%%%%%%%%%%%%%%%%%%%%%%%%%%%%%%%%%%%%%%%%%%%%%%%%%%%
\hypersetup{colorlinks=false, linkcolor=black, breaklinks=true, bookmarksdepth=3,unicode=true,bookmarksnumbered=true,
pdftitle={Hefter f�r Englisch -- Stand: \today},
pdfauthor={Thaller Alexander},
pdfsubject={Hefter f�r das Fach Englisch aus der Vorlesung von Inman f�r das 2. Semester der Informatik im Sommersemester 2009 an der Hochschule Regensburg.},
pdfkeywords={informatik,studium,hefter,englisch,sommersemester,2009}}
%</Einstellung f�r Hyperref>%%%%%%%%%%%%%%%%%%%%%%%%%%%%%%%%%%%%%%%%%%%%%%%%%%%

%<Dokument Daten>%%%%%%%%%%%%%%%%%%%%%%%%%%%%%%%%%%%%%%%%%%%%%%%%%%%%%%%%%%%%%%
\titlehead{\includegraphics[width=34mm]{FH-Logo.pdf}}
\subject{Informatik (Bachelor) 2. Semester}
\author{Alexander Thaller}
\title{Englisch\footnote{Gefundene Fehler oder Verbesserungsvorschl�ge bitte hier im PDF kommentieren, in die Fehler und Verbesserungen Textdatei schreiben oder alternativ mir eine E-Mail schicken an \href{mailto:alexander.thaller@stud.fh-regensburg.de}{alexander.thaller@stud.fh-regensburg.de}. Vielen Dank.}}
\subtitle{Hefter des Sommersemesters 2009}
\date{stand \today}
\publishers{Aus der Vorlesung von Professor Inman}
%</Dokument Daten>%%%%%%%%%%%%%%%%%%%%%%%%%%%%%%%%%%%%%%%%%%%%%%%%%%%%%%%%%%%%%

\begin{document}
\begin{acronym}
	\acro{acr}{Akronym}
	\acro{zB}[\ensuremath{\mbox{z.\,B.}\xspace}]{zum Beispiel}
	\acro{bel.}{beliebigem}
	\acro{Def.}{Definition}
	\acro{gdw.}[\ensuremath{\mbox{g.\,d.\,w.}\xspace}]{genau dann wenn}
	\acro{def.}{definiert}
	\acro{Opt.}{Optimiert}
	\acro{DEA}{Deterministischer endlicher Automat}
	\acro{akzept.}{akzeptierende}
	\acro{bzw.}{beziehungsweise}
	\acro{bzgl.}{bez�glich}
	\acro{NEA}{Nichtdeterministische endliche Automaten}
	\acro{Ber.}{Berechnung}
	\acro{Berechn.}{Berechnung}
	\acro{Bew.}{Beweis}
	\acro{d.h}[\ensuremath{\mbox{d.\,h.}\xspace}]{daher}
	\acro{akz.}{akzeptierende}
	\acro{ex.}{existiert}
	\acro{Anw.}{Anwendung}
	\acro{M�gl.}{M�glichkeit}
\end{acronym}

\maketitle
\setcounter{tocdepth}{1}
\setcounter{secnumdepth}{2}
\tableofcontents
\newpage
%<Text>%%%%%%%%%%%%%%%%%%%%%%%%%%%%%%%%%%%%%%%%%%%%%%%%%%%%%%%%%%%%%%%%%%%%%%%%
\chapter{Intel chips get power boost}



\section{Solutions}
\begin{multicols}{2}
\begin{enumerate}
\item version \label{24-03-2009-BL1-A1_1}
\item chips \label{24-03-2009-BL1-A1_2}
\item handle \label{24-03-2009-BL1-A1_3}
\item video \label{24-03-2009-BL1-A1_4}
\item systems \label{24-03-2009-BL1-A1_5}
\item Pentium \label{24-03-2009-BL1-A1_6}
\item do \label{24-03-2009-BL1-A1_7}
\item handling \label{24-03-2009-BL1-A1_8}
\item range \label{24-03-2009-BL1-A1_9}
\item applications \label{24-03-2009-BL1-A1_10}
\item the \label{24-03-2009-BL1-A1_11}
\item made \label{24-03-2009-BL1-A1_12}
\item it \label{24-03-2009-BL1-A1_13}
\item attempt \label{24-03-2009-BL1-A1_14}
\item technology \label{24-03-2009-BL1-A1_15}
\item manufacturing \label{24-03-2009-BL1-A1_16}
\item type \label{24-03-2009-BL1-A1_17}
\item it \label{24-03-2009-BL1-A1_18}
\item levels \label{24-03-2009-BL1-A1_19}
\end{enumerate}
\end{multicols}

\section{Industrial action}
\begin{multicols}{2}
\begin{enumerate}
\item to
\item the
\item immediately
\item had
\item stressing
\item chip
\item make
\item conductor
\item new
\item the
\item current
\item ensure
\item smoothly
\end{enumerate}
\end{multicols}

\chapter{Blatt 2}
\begin{multicols}{2}
\begin{enumerate}
\item giant
\item modify
\item leak
\item cram
\item alteration
\item supersede
\item collectively
\item review
\item steadily
\item accelerate
\end{enumerate}
\end{multicols}
\begin{enumerate}
\item very big and dominant (A very large company with a great market share / which is dominant in it's sector)
\item to change an existing thing (to make changes to part of a process, system \ac{etc.} without changing the whole process)
\item to slowly loose things from an containment (unwanted/unintended emission from a closed system)
\item to insert the same amount of something into a smaller place with very few room (to put/push/force more and more items into a small space)
\item an alternative version of something or doing something with some few changes
\item to make something better or faster than previous versions (replace the old version)\\
		\Ra The USB quickly superseded the parallel and RS232 ports
\item to target a group of things (together/as a group)
\item to watch and use something and rate it after specific criteria
\item to not change something over a period of time
\item to make something faster than before (increase)
\end{enumerate}

\subsection{Questions}
\begin{enumerate}
\item Because the old chips, which were used in desktops, weren't able to handle the amount of data which is used in multimedia applications. (Because image editing and multimedia applications need greater CPU\\$\begin{cases}\text{speed}\\\text{performance}\end{cases}$.)
\item The silicon becomes a better conductor. (The silicon becomes a better conductor\\$\begin{cases}\text{which makes the chip more efficient}\\\text{which increases processing speed}\end{cases}$.)
\item The loss of power. (This can lead to loss of data, processing errors, loss of performance generally and a shorter processor lifespan.)
\item The process is much larger and so has more space for memory. (The decreased width of components from 130nm to 90nm.)
\item There is older software which isn't using the advantage of the new instructions of the processor. (Because current applications have not been designed for the new chips (are not yet compatible with\dots).)
\item It allows the use of new processors with the old chip sets/motherboards. (Intel can produce the new chips without changing the production line, which saves a large amount of money.)
\end{enumerate}

\section{20.04.2009}
\begin{enumerate}
\item higher capacity (The higher capacity. - The times greater capacity.)
\item 51\%
\item They can securely disposed with simple tools (Standard DVDs cannot be disposed of easily, and sensitive data can be retrieved from discs which people throw away.)
\item In the near future (THis is not known. - This has not been announced yet.)
\item They use a blue laser instead of a red one (Paper discs will use blue laser technology, where current conventional DVDs use red laser.)
\item 25GB
\item 4.7GB
\item Because paper discs cannot hold an substrate which is needed by the red laser technology. (Red laser light needs to pass through a substrate to read and write data. Paper cannot easily bond with a substrate. Red laser light would set fire to the paper.)
\item They are cheaper in production because of the missing substrate and the cheapness of paper. (They can be produced more cheaply.)
\item They they say that they don't know what the practical uses could be. (They say they don't know.)
\end{enumerate}

\section{How should you write technical documents}
\begin{enumerate}
\item The language is impersonal
		\begin{itemize}[label=\ra]
		\item Do not use ''I'', do not use ''you''
		\item If a human agent is part of the process, then the function is used e.g. ''the user'', ''the operator'', ''the administrator'', ''the programmer'' etc.
		\item 
		\end{itemize}
\item 
\item 

\end{enumerate}
%</Text>%%%%%%%%%%%%%%%%%%%%%%%%%%%%%%%%%%%%%%%%%%%%%%%%%%%%%%%%%%%%%%%%%%%%%%%

%Literaturverzeichnis
\newpage
\addcontentsline{toc}{part}{Literaturverzeichnis}
\bibliography{literatur}{}

%Stichwortverzeichnis
\newpage
\renewcommand{\indexname}{Stichwortverzeichnis}
\addcontentsline{toc}{part}{Stichwortverzeichnis}
\printindex

\end{document}


