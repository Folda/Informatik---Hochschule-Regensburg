\chapter{2. �bung}
\section{1. Aufgabe (2+2 Punkte)}
\begin{enumerate}[label=(\alph*)]
\item Man kann den Graphen \vref{fig:Uebung_GraphA1} in Form einer Adjazentenmatrix (Abbildung \vref{fig:Uebung_A1}) schreiben welche die Aussage enth�lt ($0$ ist nicht verbunden, $1$ ist verbunden) ob zwei Knoten �ber eine Kante miteinander verbunden sind.
		\begin{figure}[htb]
		\centering
		\begin{tikzpicture}[->,>=stealth',shorten >=1pt,auto,node distance=2.8cm,semithick]
		\node[state] (n3) {3};
		\node[state] (n2) [above left of=n3] {2};
		\node[state] (n4) [below left of=n3] {4};
		\node[state] (n1) [left of=n2] {1};
		\node[state] (n5) [left of=n4] {5};

		\path[-] (n1) edge (n2);
		\path[-] (n1) edge (n5);
		
		\path[-] (n2) edge (n5);
		\path[-] (n2) edge (n3);
		\path[-] (n2) edge (n4);
		
		\path[-] (n4) edge (n5);
		\path[-] (n4) edge (n3);
		\end{tikzpicture}
		\captionsetup{labelformat=simplegraph}
		\caption{}
		\label{fig:Uebung_GraphA1}
		\end{figure}
		wird zu
		\begin{figure}[htb]
		\centering
		\[\begin{array}{c|ccccc}
		 &1&2&3&4&5\\
		\hline
		1&0&1&0&0&1\\
		2&1&0&1&1&1\\
		3&0&1&0&1&0\\
		4&0&1&1&0&1\\
		5&1&1&0&1&0
		\end{array}\]
		\caption{}
		\label{fig:Uebung_A1}
		\end{figure}

\item $w = 110 110 111 100 101 111$\\
		Alte Darstellung: $11\sharp 01\sharp 10\sharp 11\sharp 11\sharp 00\sharp 10\sharp 11\sharp 11$ (26 Zeichen)\\
		Neue Darstellung: $110\sharp 110\sharp 111\sharp 100\sharp 101\sharp 111$ (23 Zeichen)
\end{enumerate}

\section{2. Aufgabe (8+2 Punkte)}
\begin{enumerate}[label=(\alph*)]
\item $\gklamm{a}^* \gklamm{b}^* = \gklamm{a^i b^j \vert i, j \in \N}$
		\begin{align*}
		&x \in \gklamm{a}^*\gklamm{b}^*\\
		&x = y z (y \in \gklamm{a}^*, z \in \gklamm{b}^*)\\
		\Ra& y \in \gklamm{a}^k, z \in \gklamm{b}^m \tx{ f�r } \exists k, m \in \N\\
		\Ra& x = a^k b^m\\
		\Ra& x \in \gklamm{a^i b^j \vert i, j \in \N}
		\end{align*}
		\Ra $x \in \gklamm{a}^* \gklamm{b}^* \und x \in \gklamm{a^i b^j \vert i, j \in \N}$\\
		\Ra Die Aussage ist wahr

\item $\rklamm{\gklamm{a}^* \gklamm{b}^*}^* = \rklamm{\gklamm{a, b}^2}^*$\\
		$a = 0$, $b = 1$\\
		\begin{align*}
		\rklamm{\gklamm{a}^* \gklamm{b}^*}^* &= \rklamm{\gklamm{\epsilon, 0, 00, 000, \dots} \gklamm{\epsilon, 1, 11, 111, \dots}}^*\\
		&=\rklamm{\gklamm{\epsilon, 1, 11, 111, 0, 01, 011, 0111, 001, 0011, 00111, \dots}}^*\\
		&=\gklamm{\epsilon, 1, 11, 111, 0, 00, 000, 11, 1111, 111111, \dots}\\
		\rklamm{\gklamm{a, b}^2}^* &= \rklamm{\gklamm{01} \gklamm{01}}^*\\
		&= \rklamm{\gklamm{0101}}^*\\
		&= \gklamm{\epsilon, 0101, 01010101, \dots}
		\end{align*}
		\Ra $1 \in \rklamm{\gklamm{a}^* \gklamm{b}^*}^* \und 1 \notin \rklamm{\gklamm{a, b}^2}^*$\\
		\Ra Die Aussage ist falsch
\end{enumerate}

\section{3. Aufgabe (6+4+4 Punkte)}
\begin{enumerate}[label=(\alph*)]
\item $L_2 = \gklamm{w \in \Sigma_{\tx{Bool}}^* \vert \betrag{w}_0 = 4}$ - alle Bin�rw�rter, die exakt 4 Nullen enthalten.\\
		L�sung siehe Abbildung \vref{fig:Uebung2_3a}.
		\begin{figure}[hbt]
		\centering
		\begin{tikzpicture}[->,>=stealth',shorten >=1pt,auto,node distance=2.8cm,semithick]
		\node[state,initial,initial text=] (n1) {};
		\node[state] (n2) [right of=n1] {};
		\node[state] (n3) [right of=n2] {};
		\node[state] (n4) [right of=n3] {};
		\node[state,accepting] (n5) [right of=n4] {};
		\node[state] (n6) [below of=n5] {DS};
		
		\path[->] (n1) edge[loop above] node[right] {$1$} ();
		\path[->] (n2) edge[loop above] node[right] {$1$} ();
		\path[->] (n3) edge[loop above] node[right] {$1$} ();
		\path[->] (n4) edge[loop above] node[right] {$1$} ();
		\path[->] (n5) edge[loop above] node[right] {$1$} ();
		\path[->] (n6) edge[loop below] node[right] {$1, 0$} ();
		
		\path[->] (n1) edge node[above] {$0$} (n2);
		\path[->] (n2) edge node[above] {$0$} (n3);
		\path[->] (n3) edge node[above] {$0$} (n4);
		\path[->] (n4) edge node[above] {$0$} (n5);
		\path[->] (n5) edge node[right] {$0$} (n6);
		\end{tikzpicture}
		\caption{L�sung f�r 3. Aufgabe (a)}
		\label{fig:Uebung2_3a}
		\end{figure}

\item $L_3 = \gklamm{w \in \Sigma_{\tx{Bool}}^* \vert 110 \tx{ ist kein Teilwort von } w}$
		L�sung siehe Abbildung \vref{fig:Uebung2_3b}.
		\begin{figure}[hbt]
		\centering
		\begin{tikzpicture}[->,>=stealth',shorten >=1pt,auto,node distance=2.8cm,semithick]
		\node[state,initial,initial text=,accepting] (n1) {};
		\node[state,accepting] (n2) [right of=n1] {};
		\node[state,accepting] (n3) [right of=n2] {};
		\node[state] (n4) [right of=n3] {DS};

		\path[->] (n1) edge[loop above] node[right] {$0$} ();
		\path[->] (n4) edge[loop above] node[right] {$1, 0$} ();

		\path[->] (n1) edge node[above] {$1$} (n2);
		\path[->] (n2) edge node[below] {$1$} (n3);
		\path[->] (n3) edge node[above] {$0$} (n4);

		\path[->] (n2) edge [bend left] node[below] {$0$} (n1);
		\path[->] (n3) edge [bend right] node [above] {$1$} (n2);
		\end{tikzpicture}
		\caption{L�sung f�r 3. Aufgabe (b)}
		\label{fig:Uebung2_3b}
		\end{figure}

\item $L_4 = \gklamm{w \in \Sigma_{\tx{Bool}}^* \vert \betrag{w} \klgl 5}$
		L�sung siehe Abbildung \vref{fig:Uebung2_3c}.
		\begin{figure}[hbt]
		\centering
		\begin{tikzpicture}[->,>=stealth',shorten >=1pt,auto,node distance=2.8cm,semithick]
		\node[state,initial,initial text=,accepting] (n1) {};
		\node[state,accepting] (n2) [right of=n1] {};
		\node[state,accepting] (n3) [right of=n2] {};
		\node[state,accepting] (n4) [right of=n3] {};
		\node[state,accepting] (n5) [right of=n4] {};
		\node[state,accepting] (n6) [right of=n5] {};
		\node[state] (n7) [below of=n6] {DS};

		\path[->] (n7) edge [loop below] node[right] {$1,0$} ();
		\path[->] (n1) edge node[above] {$1, 0$} (n2);
		\path[->] (n2) edge node[above] {$1, 0$} (n3);
		\path[->] (n3) edge node[above] {$1, 0$} (n4);
		\path[->] (n4) edge node[above] {$1, 0$} (n5);
		\path[->] (n5) edge node[above] {$1, 0$} (n6);
		\path[->] (n6) edge node[right] {$1, 0$} (n7);
		\end{tikzpicture}
		\caption{L�sung f�r 3. Aufgabe (c)}
		\label{fig:Uebung2_3c}
		\end{figure}
\end{enumerate}

\section{4. Aufgabe (3 Punkte)}
\[\left.
\begin{array}{lll}
h(0) &=& 000\\
h(1) &=& 001\\
h(x) &=& 010\\
h('(') &=& 011\\
h(')') &=& 100\\
h(\vee) &=& 101\\
h(\wedge) &=& 110\\
h(\neg) &=& 111\\
\end{array}\right\} h: \Sigma_{\tx{logic}}^* \ra \Sigma_{\tx{Bool}}^*
\]

%\section{5. Aufgabe (10 Zusatzpunkte)}
