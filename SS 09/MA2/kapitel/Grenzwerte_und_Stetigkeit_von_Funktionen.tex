\chapter{Grenzwerte und Stetigkeit von Funktionen}
\Mark{Chapter 3.1}
Wir betrachten eine Funktion $f$: $D \ra \R$ in einer Umgebung des Punktes $x_0 \in D$.

\begin{fdefinition}[Grenzwerte]
\mbox{}\par
\begin{enumerate}[label=\alph*)]
\item Die Funktion $f$ hat im Punkt $x_0$ den \indexu{Grenzwert} $a$, $\lim_{x \ra x_0} f(x) = a$, falls gilt:
		\[\forall(x_n)_{n \in \N} \tx{ mit } x_n \in D, \lim_{n \ra \un} x_n = x_0:~\lim_{n \ra \un} f(x_n) = a\]

\item Die Funktion $f$ hat im Punkt $x_0$ den \indexu{linksseitigen Grenzwert} $a \lim_{x \nearrow x_0} f(x) = a$ (\ac{bzw.} $\lim_{x \uparrow x_0} f (x) = a$, $\lim_{x \ra x_0^-} f(x) = a$), falls gilt:
		\[\forall(x_n)_{n \in \N} \tx{ mit } x_n < x_0, x_n \in D, \lim_{n \ra \infty} x_n ? x_0:~\lim_{n \ra \infty} f(x_n) = a\]

\item Die Funktion $f$ hat im Punkt $x_0$ den \indexu{rechtsseitigen Grenzwert} $a$ $\lim_{x \searrow x_0} f (x) = a$ (\ac{bzw.} $\lim_{x \downarrow x_0} f(x) = a$ \ac{bzw.} $\lim_{x \ra x_0^+} f (x) = a$), falls gilt
		\[\forall(x_n)_{n \in \N} \tx{ mit } x_n \in D, x_n > x_0, \lim_{n \ra \infty} x_n = x_0\]

\item Die Funktion $f$ hat in $\pm \un$ den Grenzwert $a$ $\lim_{x \ra \pm \infty} f (x) = a$, falls gilt:
		\[\forall(x_n)_{n \in \N} \tx{ mit } x_n \in D, \lim_{n \ra \infty} = \pm \infty:~\lim_{n \ra \infty} f(x_n) = a\]

\item Die Funktion $f$ hat in $x_0$ den \indexu{uneigentlichen Grenzwert} $\pm \infty$, $\lim_{x \ra x_0} f (x) = \pm \un$ falls gilt:
		\[\forall(x_n)_{n \in \N} \tx { mit } x_n \in D, \lim_{n \ra \un} x_n = x_0:~\lim_{n \ra \un} f(x_n) = \pm \un\]
		(Links-/rechtsseitige uneigentliche Grenzwert werden analog definiert).
\end{enumerate}
\Mark{Definition 3.1}
\end{fdefinition}

\begin{bemerkung}
Es gilt $\lim_{x \ra x_0} f(x) = a \Lra \lim_{x \nearrow x_0} f(x) = \lim_{x \searrow x_0} f(x) = a$
\end{bemerkung}

\begin{fsatz}[Rechenregeln f�r Grenzwerte]
\label{satz:Grenzwerte_und_Stetigkeit_von_Funktionen_S3_2}
Falls $\lim_{x \ra x_0} f (x) = a$ und $\lim_{x \ra x_0} g (x) = b$ ($a, b \in \R$) gilt:
\begin{enumerate}[label=\roman*)]
\item $\lim_{x \ra x_0} (f(x) \pm g(x)) = \lim_{x \ra x_0} f(x) \pm \lim_{x \ra x_0} g(x) = a \pm b$
\item $\lim_{x \ra x_0} (f(x) \mal g(x)) = \lim_{x \ra x_0} f(x) \mal \lim_{x \ra x_0} g(x) = a \mal b$
\item $\lambda \in \R$: $\lim_{x \ra x_0} \lambda \mal f(x) = \lambda \mal \lim_{x \ra x_0} f(x) = \lambda \mal a$
\item falls $b \neq 0$: $\lim_{x \ra x_0} \frac{f(x)}{g(x)} = \frac{\lim_{x \ra x_0} f(x)}{\lim_{x \ra x_0} g(x)} = \frac{a}{b}$
\end{enumerate}
\Mark{Satz 3.2}
\end{fsatz}

\begin{beweis}
Folgt aus Rechenregeln f�r Grenzwerte von Folgen.
\end{beweis}

\begin{bemerkung}
Satz \vref{satz:Grenzwerte_und_Stetigkeit_von_Funktionen_S3_2} gilt analog f�r links-/rechtsseitige Grenzwerte.
\Solved{Referenziere zu Satz 3.2}{}
\end{bemerkung}

\begin{fdefinition}[Stetigkeit]
Eine Funktion $f$: $D \ra \R$ hei�t stetig in $x_0 \ra D$, wenn
\[\lim_{x \ra x_0} f(x) = f(x_0)\]
$f$ hei�t stetig in der Menge $A \subseteq D$, wenn $f$ in jedem Punkt $x_0 \in A$ stetig ist.
\Mark{Definition 3.3}
\end{fdefinition}
