\chapter{H�here Ableitungen und Taylor-Entwicklung}
\Mark{Chapter 4.4}
\begin{fdefinition}
Die Funktion $f$ hei�t $n$-mal differenzierbar falls $f' \frac{d}{dx} f$, $f'' \frac{d}{dx} f'$, $f''' \frac{d}{dx} f''$, $f^{(n)} \frac{d}{dx} f^{(n - 1)}$ existieren.
\Mark{Definition 4.17}
\end{fdefinition}

\begin{bemerkung}
statt $f^{(n)}$ kann man auch schreiben
\[\frac{d^n}{dx^n} f \rkl{=\frac{d}{dx} \mal \frac{d}{dx} \mal \dots \mal \frac{d}{dx} f}\]
\end{bemerkung}

\begin{fsatz}[Leibniz-Produktregel]
Die Funktionen $f$ und $g$ seien $n$-mal differenzierbar. Dann ist auch das Produkt $f \mal g$ $n$-mal differenzierbar und es gilt:
\[(f \mal g)^{(n)} (x) = \sum_{k = 0}^{n} \binom{n}{k} \mal f^{(k)} (x) \mal g^{(g - k)} (x)\]
\end{fsatz}

\begin{fbeweis}[Beweisskizze]
zu zeigen:
\begin{enumerate}
\item Existenz der $n$-ten ABleitung von $f \mal g$
\item Darstellung der Ableitung durch obige Summe
\end{enumerate}
zuerst 2. zeigen mit vollst�ndiger Induktion (\ac{d.h.} falls \ggq{Ableitung} existiert hat sie diese Form)\\
da alle Summanden der rechten Seite existieren \Ra Summe existiert \ra $(f \mal g)^{(n)}$ existiert.

\begin{beispiel}
$h(x) = \ub{x^2}{f(x)} \mal \ub{\cosx}{g(x)}$ gesucht $h^{(5)}$
\begin{align*}
&f(x) = x^2 & g(x) = \cosx\\
&f'(x) = 2x & g'(x) = -\sinx\\
&f''(x) = 2 & g''(x) = -\cosx\\
&f'''(x) = 0 & g'''(x) = \sinx\\
&f^{(4)}(x) = 0 & g^{(4)}(x) = \cosx\\
&f^{(5)}(x) = 0 & g^{(5)}(x) = -\sinx
\end{align*}
\begin{align*}
\rkl{f \mal g}^{(5)} (x) &= \sum_{k = 0}^{5} \binom{5}{k} f^{(k)} (x) \mal g^{(n - k)} (x)\\
&= \binom{5}{0} \mal x^2 \mal \rkl{- \sinx} + \binom{5}{1} \mal 2x \cosx + \binom{5}{2} \mal 2 \mal \sinx\\
&+ \binom{5}{3} \mal 0 \mal \rkl{- \cosx} + \binom{5}{4} \mal 0 \mal \rkl{-\sinx} + \binom{5}{5} \mal 0 \cosx\\
&= - x^2 \sinx + 10x \mal \cosx + 20 \sinx\\
&= \sinx \mal \rkl{20 - x^2} + 10x \cosx
\end{align*}
\end{beispiel}
\end{fbeweis}

\section{Aufgabe 4.13}
\label{sec:Hoehere_Ableitungen_und_Taylor_Entwicklung_A4_13}
Es soll die FUnktion $f(x) = \arcsin \rkl{x^2 -1}$ betrachtet werden.
\begin{enumerate}[label=\alph*)]
\item Bestimmen Sie den maximalen Definitionsbereich der Funktion $f$.
\item Berechnen Sie die Ableitung $f'(x)$ und deren Definitionsbereich.
\end{enumerate}

L�sung siehe \vref{sec:Hoehere_Ableitungen_und_Taylor_Entwicklung_A4_13L}.

\section{Reihendarstellung}
F�r eine oft differenzierbare Funktion h�tte man gerne eine Darstellung durch eine Potenzreihe oder zumindest eine N�herung durch eine endliche Reihe.

\subsection{Betrachten Potenzreihen}
\begin{align*}
f(x) &= \sum_{k = 0}^{\un} a_k \mal (x - x_0)^k\\
f(x_0) &= \sum_{k = 0}^{\un} a_k \mal \ub{(x_0 - x_0)}{0}^k = a_0 \Ra a_0 = f(x_0)\\
f'(x) &= \sum_{k = 1}^{\un} k \mal a_k \mal (x - x_0)^{k - 1}\\
f'(x_0) &= \sum_{k = 1}^{\un} k \mal a_k \mal 0^{k - 1} = 1 \mal a_1 \Ra a_1 = f'(x_0)\\
f''(x) &= \sum_{k = 2}^{\un} k \mal (k - 1) \mal a_k \mal (x - x_0)^{k - 2}\\
f''(x_0) &= \sum_{k = 2}^{\un} a_k \mal 0^{k - 2} = 2 \mal 1 \mal a_2 \Ra a_2 = \frac{f''(x_0)}{2 \mal 1}\\
f^{(n)} &= \sum_{k = n}^{\un} k \mal (k - 1) \mal \dots \mal (k - n + 1) \mal a_k (x - x_0)^{k - n}\\
&= \sum_{k = n}^{\un} \frac{k!}{(k - n)!} a_k \mal (x - x_0)^{k - n}\\
f^{(n)} (x_0) &= \sum_{k = n}^{\un} \frac{k!}{(k - n)!} \mal a_k \mal 0^{k - n} = \frac{n!}{0!} \mal a_n
\Ra& a_n =\frac{f^{(n)} (x_0)}{n!}
\end{align*}

\begin{fdefinition}[Taylor-Polynom]
F�r eine $n$-mal differenzierbare Funktion $f$ hei�t
\[T_n(f,x) = \sum_{k = 0}^{n} \frac{f^{(k)} (x_0)}{k!} \mal (x - x_0)^{k}\]
das $n$-te \indexb{Taylor-Polynom} von $f$ um $x_0$
\Mark{Definition 4.19}
\end{fdefinition}

\begin{fsatz}[Satz von Taylor]
Die Funktion $f$: $\eklamm{a, b} \ra \R$ sei $(n + 1)$-mal stetig differenzierbar (\ac{d.h.} $f$ ist $(n + 1)$-mal differenzierbar und $f^{(n + 1)}$ ist stetig) und $x_0 \in (a,b)$\\
Dann gilt f�r alle $x \in \eklamm{a, b}$ die Taylor-Formel
\[f(x) = \sum_{k = 0}^{n} \rac{f^{(k)} (x_0)}{k!} (x - x_0)^{k} + R_{n + 1} (x)\]
und es existiert ein $\xi$ zwischen $x$ und $x_0$ so dass
\[R_{n + 1} (x) = \frac{f^{(n + 1)} (\xi)}{(n + 1)!} \mal (x - x_0)^{n + 1}\]
\indexb{Lagrange-Form des Restglieds}
\Mark{Satz 4.20}
\end{fsatz}

\begin{bemerkung}
Manchmal wird das zum Taylor-Polynom $n$-ter Ordnung geh�rende Restglied auch mit $R_n$ (statt $R_{n + 1}$) bezeichnet.
\end{bemerkung}

\begin{fbeweis}
-
\end{fbeweis}

\subsection[Spezialf�lle}
\begin{enumerate}
\item $n = 0$
		\begin{align*}
		f(x) &= T_0 (f, x) + R_1 (x)\\
		&= f(x_0) + \frac{f'(\xi)}{1!} \mal (x - x_0)^1
		\end{align*}
		mit $\xi$ zwischen $x$ und $x_0$\\
		Wir formen um zu
		\begin{align*}
		&f(x) - f(x_0) = f'(\xi) \mal (x - x_0)\\
		\Lra &\frac{f(x) - f(x_0)}{x - x_0} = f'(\xi) \ra \tx{Mittelwertsatz der Differentialrechnung}
		\end{align*}

\item $n = 1$
		\begin{align*}
		f(x) &= T_1(f, x) + R_2 (x)\\
		&=f (x_0) + f'(x_0) \mal (x - x_0) + R_2 (x)
		\end{align*}
		Das Taylorpolynom $T_1(f, x)$ stellt die lineare Approximation der Funktion $f$ durch die Tangente in $x_0$ dar.

\item $n = 2$
		\begin{align*}
		f(x) &= T_2(f, x) + R_3 (x)\\
		&=f (x_0) + f'(x_0) \mal (x - x_0) + \frac{f'' (x_0)}{2} \mal (x - x_0)^2 + R_3 (x)
		\end{align*}
		$T_2(f,x)$ ist eine quadratische N�herung der Funktion $f$, also eine N�herung durch eine Parabel.
\end{enumerate}

\begin{fdefinition}[Taylorreihe]
Ist eine Funktion beliebig oft differenzierbar, so hei�t die Potenzreihe
\[T(f, x) = \sum_{k = 0}^{\un} \frac{f^{(k)} (x_0)}{k!} (x - x_0)^k\]
die Taylorreihe von $f$ mit Entwicklungspunkt $x_0$
\end{fdefinition}

\begin{bemerkung}
\fnewline
\begin{enumerate}
\item Es gibt keine allgemeinen Aussagen �ber die Konvergenz von Taylorreihen (also ob ein Konvergenzradius $r > 0$ existiert und wie gro� er gegebenenfalls ist).
\item Im Fall der Konvergenz der Taylorreihe (also innerhalb des Konvergenzgebietes) gilt nicht immer $f(x) = T(f, x)$.
\item Funktionen deren Taylorreihe konvergiert und f�r die im Konvergenzgebiet gilt $f(x) = T(f, x)$ nennt man \indexb{analytisch}. Viele gebr�uchliche Funktionen sind analytisch: $e^x$, $\lnx$, $\sinx$, $\cosx$, $\tanx$, $\frac{1}{(1 - x)}$, $\rkl{\frac{1}{1 - x}}^2$, $\arcsin (x)$ \dots
\end{enumerate}
\end{bemerkung}

\begin{beispiel}
\fnewline
\begin{enumerate}
\item $f(x) = \lnx$, $x_0 = 1$\\
		Gesucht Taylorreihe mit Entwicklungspunkt $x_0$
		\begin{align*}
		f(x) = \lnx &\Ra f(1) = \lnx[1] = 0\\
		f'(x) = \frac{1}{x} = x^{-1} &\Ra f'(1) = \frac{1}{1} = 1\\
		f''(x) = (-1) \mal x^{-2} &\Ra f''(1) = (-1)\\
		f'''(x) = (-1) (-2) x^{-3} &\Ra f'''(1) = 2 \mal x^{-3}\\
		f^{(4)} (x) = (-1) \mal (-2) \mal (-3) \mal x^{-4} &\Ra f^{(4)} (1) = (-3) \mal 2\\
		f^{(k)} (x) = x^{-k} \mal (k - 1)! \mal (-1)^{k - 1} &\Ra f^{(k)} (1) = (-1)^{k - 1} \mal (k - 1)!
		\end{align*}
		(k�nnte man mit vollst�ndiger Induktion beweisen)
		\[T(f, x) = \sum_{k = 0}^{\un} \frac{f^{(k)} (1)}{k!} (x - 1)^k = \sum_{k = 1}^{\un} (-1)^{k - 1} \mal \frac{(k - 1)!}{k!} \mal (x - 1)^k\]
		\[= \sum_{k = 1}^{\un} \frac{(-1)^{k}{k} (x - 1)^k\]
		Konvergenzgebiet
\end{enumerate}
\end{beispiel}

\begin{beispiel}
Taylorreihe f�r $\lnx = f(x)$, $x_0 = 1$
\begin{align*}
&\vdots\\
&T \rkl{\lnx, x} = \sum_{k = 1}^{\un} \ub{(-1)^{k + 1} \mal \frac{1}{k}}{a_k} (x - 1)^k
\end{align*}
Konvergenzradius
\begin{align*}
w &= \lim_{k \ra \un} \sqrt[k]{\betrag{a_k}} = \lim_{k \ra \un} \sqrt[k]{\betrag{(-1)^{k + 1} \frac{1}{k}}}\\
&= \lim_{k \ra \un} \sqrt[k]{\frac{1}{k}} = 1
\end{align*}
\Todo{Von Wolf nachtragen (26.05.2009)}
\end{beispiel}

\Todo{Nachtragen von 27.05.2009}

\subsection{Bestimmung von globalen Extrema auf offenen Intervallen}
$f$: $(a, b) \ra \R$ differenzierbar mit $a \in \eklamm{-\un, \un}$ $b \in (-\un, \un]$

\subsubsection{Vorgehensweise}
\begin{enumerate}
\item Bestimmen aller station�rer Punkte $x_1, x_2, \dots, x_e$
\item Berechnen von $f(x_1), f(x_2), \dots, f(x_e)$\\
		$\lim_{x \searrow a} f(x), \lim_{x \nearrow b} f(x)$

\item Ist $\max \rklamm{f(x_1), f(x_2), \dots, f(x_e), \lim_{x \searrow a} f(x), \lim_{x \nearrow b} f(x)} \in \eklamm{\lim_{x \searrow a} f(x), \lim_{x \nearrow b} f(x)}$ dann gibt es kein globales Maximum\\
		Andernfalls Maximalwert = $\max \rkl{f(x_1), f(x_2), \dots, f(x_e)}$ f�r Minimum analog\\
		\Insert{MA2-28.05.2009-IMG-1} hat kein globales Maximum auf Intervall $(0, 2)$ oder auch auf Intervall $(1, 2)$
		\begin{bemerkung}
		Es ist nicht immer n�tig \ac{bzw.} nicht immer m�glich $\lim_{x \searrow a} f(x)$, $\lim_{x \nearrow b} f(x)$ zu berechnen.
		\begin{enumerate}
		\item[nicht n�tig] zum Beispiel $f$ hat nur in $x_1$ einen station�ren Punkt. Wenn dann nur das globale Maximum gesucht ist und wir zeigen k�nnen, dass in $x_1$ ein lokales Maximum liegt ist in $x_0$ auch das globale Maximum.\\
				Falls wir das globale Maximum suchen und in $x_1$ ist ein lokales Minimum, dann existiert kein globales Maximum.
				\Insert{MA2-28.05.2009-IMG-2}

		\item[nicht m�glich] \Insert{MA2-28.05.2009-IMG-3}
				dann existiert $\lim_{x \ra \un} f(x)$ nicht
		\end{enumerate}
		\end{bemerkung}
\end{enumerate}

\section{Aufgabe 4.5}
\label{sec:Hoehere_Ableitung_und_Taylor_Entwicklung_A4_5}
Finde den Punkt auf der Parabel $y^2 = 2x$ der am n�hesten von $(1, 4)$ liegt.

L�sung siehe \vref{sec:Hoehere_Ableitung_und_Taylor_Entwicklung_A4_5L}.

\section{Newton Verfahren}
\Mark{2)??}
Bestimmung von Nullstellen einer differenzierbaren Funktion
\Insert{MA2-28.05.2009-IMG-1}
Tangente an $f$ in $x_0$
\[t_0 (x) = f(x_0) + f'(x_0) \mal (x - x_0)\]
mit der $x$-Achse schneiden (\Ra $y$-Wert ist $0$)\\
Schnittpunkt $x_1$

\section{L�sungen}
\subsection{Aufgabe 4.13}
\subsection{Aufgabe 4.6}
\label{sec:Hoehere_Ableitungen_und_Taylor_Entwicklung_A4_13L}
L�sung zu Aufgabe \vref{sec:Hoehere_Ableitungen_und_Taylor_Entwicklung_A4_13}.

$f(x) = \arcsin \rkl{x^2 - 1}$
\begin{enumerate}[label=\alph*)]
\item Definitionsbereich von $\arcsin(y)$ ist $\eklamm{-1, 1}$
		\begin{align*}
		&-1 \klgl x^2 - 1 \klgl 1 \Vert +1\\
		\Lra &0 \klgl x^2 \klgl 2\\
		\Lra &\ub{0 \klgl x^2}{\tx{gilt immer}} \und x^2 \klgl 2\\
		\Lra &x^2 \klgl 2\\
		\Lra &-\sqrt{2} \klgl x \klgl \sqrt{2}\\
		\Ra & D = \eklamm{-\sqrt{2}, \sqrt{2}}
		\end{align*}

\item \begin{align*}
		\frac{d}{dx} \arcsin \rkl{x^2 - 1} &= \frac{1}{\sqrt{1 - \rkl{x^2 - 1}^2}} \mal 2x\\
		&= \frac{2}{\sqrt{1 - x^4 + 2x^2 - 1}} = \frac{2x}{\sqrt{x^2} \mal \sqrt{2 - x^2}}\\
		&= \begin{cases}\frac{2}{\sqrt{2 - x^2}} & \tx{f�r } x > 0\\\frac{-2}{\sqrt{2 - x^2}} & \tx{f�r } x < 0\end{cases}
		\end{align*}
		Definitionsbereich der Ableitung\\
		$\rkl{-\sqrt{2}, 0} \cup \rkl{0, \sqrt{2}}$
\end{enumerate}

\subsection{Aufgabe 4.5}
\subsection{Aufgabe 4.6}
\label{sec:Hoehere_Ableitung_und_Taylor_Entwicklung_A4_5L}
L�sung zu Aufgabe \vref{sec:Hoehere_Ableitung_und_Taylor_Entwicklung_A4_5}.

$y^2 = 2x$

\Insert{MA2-28.05.2009-IMG-3}

\begin{align*}
A(y) &= \sqrt{\rkl{\frac{1}{2} y^2 - 1}^2 + (y - 4)^2}\\
f(y) &= \rkl{\frac{1}{2} y^2 - 1}^2 + (y - 4)^2\\
f'(y) &= 2 \mal \rkl{\frac{1}{2} y^2 - 1} \mal \rkl{\frac{1}{2} \mal 2 \mal y} + 2 (y - 4)\\
&= x^3 - 2y + 2y - 8 = y^3 - 8
\end{align*}

\begin{align*}
&f'(y_0) \stack{!}{=} 0\\
\Lra & y_0^3 - 8 = 0\\
\Lra & y_0^3 = 8\\
\Lra & y_0 = 2\\
&\Ra x_0 = \frac{1}{2} y_0^2 = 2\\
& f'' (y) = 3y^2 > 0\\
\end{align*}
\Ra in $(2,2)$ ist das globale MInimum, also der Punkt der Parabel der am n�chsten zu $(1,4)$ liegt.
